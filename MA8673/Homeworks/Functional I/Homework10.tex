\documentclass[]{article}

% Packages for mathematics
\usepackage{amsmath, amssymb, amsthm}
\usepackage{mathtools}
\usepackage{gensymb}
\usepackage{bm}
\usepackage{authblk}

% Package for graphics
\usepackage{graphicx}
\graphicspath{{../figures/}}
\usepackage{subcaption}
\usepackage{placeins}


% Package for page layout and headers/footers
%\usepackage{geometry}
%\geometry{margin=1in}


% Package for clickable links
\usepackage{hyperref}
\hypersetup{
    colorlinks=true,
    linkcolor=blue,
    citecolor=blue,
    urlcolor=blue,
}

% Package for algorithms
\usepackage{float}
\usepackage{algorithm}
\usepackage{algpseudocode}
\usepackage{matlab-prettifier}
\usepackage{tabularx}
\usepackage{amsmath, physics}

% Custom theorem environments
\newtheorem{theorem}{Theorem}[section]
\newtheorem{lemma}[theorem]{Lemma}
\newtheorem{corollary}[theorem]{Corollary}
\newtheorem{proposition}{Proposition}
\newtheorem{remark}{Remark}
\theoremstyle{definition}
\newtheorem{definition}[theorem]{Definition}

\title{Functional Analysis Homework 10}
\author{Kevin Ho}
\date{\today}

\begin{document}

\maketitle

\begin{enumerate}
    \item Show that all finite-dimensional normed spaces $X$ have Schur property, so the weak and strong convergence in $X$ coincide. 
    \begin{proof}
        Let $X$ be finite dimensional and $x_n \rightharpoonup x$. Weak convergence implies boundedness, so ${x_n}$ lies in a closed ball. In finite dimensions, closed and bounded sets are compact, hence ${x_n}$ has a norm-convergent subsequence $x_{n_k}\to y$. Norm convergence $\Rightarrow$ weak convergence, so $x_{n_k}\rightharpoonup y$. Weak limits are unique, thus $y=x$.
        
        In a metric space, if every subsequence has a further subsequence converging to $x$, then the whole sequence converges to $x$. Therefore $x_n\to x$ in norm. Hence weak and strong convergence coincide on $X$.
    \end{proof}
    

    
    \item Prove that in an infinite dimensional normed space $X$, weak topology is stricktly weaker than the strong topology. Why does this not contradict Schur property of $X = \ell_1$ mentioned in Remark 3.5.14?
\begin{proof}
(\emph{Strictness}) Consider the open unit ball $B={x:|x|<1}$, which is norm-open. We show $B$ is \emph{not} weakly open. If it were, some basic weak neighborhood of $0$,
$$
U={x\in X:\ |f_i(x)|<\varepsilon,\ i=1,\dots,m},
$$
would satisfy $U\subset B$ for some $f_1,\dots,f_m\in X^*$ and $\varepsilon>0$. But $V:=\bigcap_{i=1}^m\ker f_i$ has finite codimension and is infinite dimensional. Thus we can choose $x\in V$ with $|x|>1$. Then $x\in U$  but $x\notin B$, a contradiction. Hence $B$ is not weakly open, so the weak topology is strictly coarser.

({No contradiction with $\ell_1$}) The Schur property is a {sequential} statement: if $x_n\rightharpoonup x$ in $\ell_1$, then $|x_n-x|\to0$. Topological equality is stronger: it requires the collections of {all} open sets to match. In infinite-dimensional spaces (including $\ell_1$) the weak topology is not first countable, so agreeing on convergent sequences does \emph{not} force the topologies to be identical. Thus $\ell_1$ can have Schur while weak $\neq$ norm topologies.
\end{proof}

    \item Show that $\ell_\infty$ is a universal space for all separable Banach spaces. In other words, show that every separable Banach space $X$ isometrically embeds into $\ell_\infty$.
\begin{proof}
Let $X$ be separable. By Banach--Alaoglu, the closed unit ball $B_{X^*}$ is weak$^*$-compact; since $X$ is separable, $B_{X^*}$ is metrizable in the weak$^*$ topology, hence has a countable dense subset $\{f_n\}_{n\ge 1}\subset B_{X^*}$.

Define $T:X\to \ell_\infty$ by
\[
T(x):=\big(f_1(x),f_2(x),\dots\big).
\]
Linearity is clear. For each fixed $x\in X$, the map $\Phi_x:B_{X^*}\to\mathbb{K}$ given by $\Phi_x(f)=|f(x)|$ is weak$^*$-continuous, so
\[
\sup_{n\ge 1}|f_n(x)|
=
\sup_{f\in \overline{\{f_n\}}^{\,w^*}} |f(x)|
=
\sup_{f\in B_{X^*}} |f(x)|.
\]
By Hahn--Banach,
\[
\|x\|=\sup_{f\in B_{X^*}}|f(x)|.
\]
Therefore
\[
\|T(x)\|
=
\sup_{n\ge 1}|f_n(x)|
=
\|x\|,
\]
so $T$ is an isometry (hence injective). Thus $X$ embeds isometrically into $\ell_\infty$.
\end{proof}



\end{enumerate}

\end{document}
