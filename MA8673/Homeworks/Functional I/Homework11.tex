\documentclass[]{article}

% Packages for mathematics
\usepackage{amsmath, amssymb, amsthm}
\usepackage{mathtools}
\usepackage{gensymb}
\usepackage{bm}
\usepackage{authblk}

% Package for graphics
\usepackage{graphicx}
\graphicspath{{../figures/}}
\usepackage{subcaption}
\usepackage{placeins}


% Package for page layout and headers/footers
%\usepackage{geometry}
%\geometry{margin=1in}


% Package for clickable links
\usepackage{hyperref}
\hypersetup{
    colorlinks=true,
    linkcolor=blue,
    citecolor=blue,
    urlcolor=blue,
}

% Package for algorithms
\usepackage{float}
\usepackage{algorithm}
\usepackage{algpseudocode}
\usepackage{matlab-prettifier}
\usepackage{tabularx}
\usepackage{amsmath, physics}

% Custom theorem environments
\newtheorem{theorem}{Theorem}[section]
\newtheorem{lemma}[theorem]{Lemma}
\newtheorem{corollary}[theorem]{Corollary}
\newtheorem{proposition}{Proposition}
\newtheorem{remark}{Remark}
\theoremstyle{definition}
\newtheorem{definition}[theorem]{Definition}

\title{Functional Analysis Homework 11}
\author{Kevin Ho}
\date{\today}

\begin{document}

\maketitle

\begin{enumerate}
    \item Show that $T$ is compact iff it maps $B_X$ to a precompact set in $Y$.


    \begin{proof}
Let $X,Y$ be normed spaces and $T:X\to Y$ linear and bounded. Write
\[
B_X:=\{x\in X:\|x\|\le 1\}.
\]
Recall that a set $M\subset Y$ is \emph{precompact} if its closure
$\overline{M}$ is compact in $Y$.

We show the two implications separately.

\medskip\noindent
($\Rightarrow$): Suppose $T$ is compact in the sense that it maps
bounded subsets of $X$ into precompact subsets of $Y$. Since $B_X$ is
bounded in $X$, it follows immediately that $T(B_X)$ is precompact in
$Y$.

\medskip\noindent
($\Leftarrow$): Now assume that $T(B_X)$ is precompact in $Y$. We must
show that $T$ maps any bounded subset of $X$ to a precompact subset of
$Y$.

Let $A\subset X$ be bounded. Then there exists $R>0$ such that
$\|x\|\le R$ for all $x\in A$, i.e.
\[
A\subset R B_X:=\{x\in X:\|x\|\le R\}.
\]
By linearity,
\[
T(RB_X)=\{T(Rx):\|x\|\le 1\}
=\{R\,T(x):\|x\|\le 1\}
=R\,T(B_X).
\]

Since $T(B_X)$ is precompact, its closure $K:=\overline{T(B_X)}$ is
compact in $Y$. Consider the map $S_R:Y\to Y$, $S_R(y)=Ry$ for the
fixed scalar $R>0$. This is a homeomorphism with continuous inverse
$S_R^{-1}(y)=\frac1R y$, so it sends compact sets to compact sets.
Hence
\[
\overline{R\,T(B_X)}
= \overline{S_R(T(B_X))}
\subset S_R(\overline{T(B_X)})
= R K
\]
is compact as a closed subset of the compact set $R K$. Therefore
$R\,T(B_X)$ is precompact in $Y$.

Now $T(A)\subset T(RB_X)=R\,T(B_X)$, and any subset of a precompact
set is again precompact (its closure is contained in the compact
closure of the larger set). Thus $T(A)$ is precompact in $Y$.

Since $A\subset X$ was an arbitrary bounded set, we have shown that
$T$ maps every bounded subset of $X$ to a precompact subset of $Y$.
By definition, this means $T$ is compact.

\end{proof}

    \item Show that Volterra operator is compact on $C[0,1],$ even though its kernel is discontinous.


    \begin{proof}
First, $T:C[0,1]\to C[0,1]$ is linear by properties of the integral. If $\|f\|_\infty\le M$, then for all $x\in[0,1]$,
\[
|(Tf)(x)| = \left|\int_0^x f(t)\,dt\right|
\le \int_0^x |f(t)|\,dt
\le M x \le M,
\]
so $\|Tf\|_\infty\le M$ and therefore $\|T\|\le 1$; in particular, $T$ is bounded.

To prove compactness, it suffices (by the previous exercise) to show that $T$ maps the unit ball
\[
B:=\{f\in C[0,1]:\|f\|_\infty\le 1\}
\]
into a precompact subset of $C[0,1]$. By Arzelà–Ascoli, it is enough to show $\{Tf:f\in B\}$ is uniformly bounded and equicontinuous.

Uniform boundedness: for $f\in B$ and $x\in[0,1]$,
\[
|(Tf)(x)| \le \int_0^x |f(t)|\,dt \le \int_0^x 1\,dt = x \le 1,
\]
so $\|Tf\|_\infty\le 1$ for all $f\in B$.

Equicontinuity: let $f\in B$ and $0\le x<y\le 1$. Then
\[
|(Tf)(y)-(Tf)(x)|
= \left|\int_0^y f(t)\,dt - \int_0^x f(t)\,dt\right|
= \left|\int_x^y f(t)\,dt\right|
\le \int_x^y |f(t)|\,dt
\le \int_x^y 1\,dt
= |y-x|.
\]
This bound does not depend on $f\in B$, so the family $\{Tf:f\in B\}$ is equicontinuous.

Thus $T(B)$ is uniformly bounded and equicontinuous, hence its closure is compact in $C[0,1]$ by Arzelà–Ascoli. Therefore $T$ maps $B$ to a precompact set, so $T$ is compact.

\end{proof}

    \item Fix a sequence of real numbers $\{\lambda_k\}^\infty_{k=1}$, and define the linear operator $T:\ell_2\rightarrow\ell_2$ by 
    $$
    Tx = \{\lambda_kx_k\}^\infty_{k=1}.
    $$
    For what multiplier sequences $\{\lambda_k\}^\infty_{k=1}$ is the operator $T$, (a) well defined? (b) bounded? (c) compact?

    \begin{proof}
(a) \emph{Well defined.}
First assume $(\lambda_k)$ is bounded, say $M:=\sup_k|\lambda_k|<\infty$. Then for $x\in\ell_2$,
\[
\|Tx\|_2^2
= \sum_{k=1}^\infty |\lambda_k x_k|^2
\le M^2 \sum_{k=1}^\infty |x_k|^2
= M^2 \|x\|_2^2 <\infty,
\]
so $Tx\in\ell_2$ and $T$ is well defined.

Conversely, suppose $T$ is well defined but $(\lambda_k)$ is unbounded. Then we can pick indices $k_j$ with
$|\lambda_{k_j}|\ge j$. Define $x\in\ell_2$ by
\[
x_{k_j} = \frac{1}{j}, \qquad x_k=0 \text{ otherwise.}
\]
Then $\|x\|_2^2=\sum_{j=1}^\infty \frac1{j^2}<\infty$, so $x\in\ell_2$. But
\[
|\lambda_{k_j}x_{k_j}|^2 \ge j^2 \cdot \frac{1}{j^2} = 1,
\]
hence
\[
\sum_{k=1}^\infty |\lambda_k x_k|^2 \ge \sum_{j=1}^\infty 1 = \infty,
\]
so $Tx\notin\ell_2$, contradicting that $T$ is well defined. Thus $(\lambda_k)$ must be bounded.
Therefore
\[
\text{$T$ is well defined} \iff (\lambda_k)\in \ell_\infty.
\]

\medskip
(b) \emph{Bounded.}
If $(\lambda_k)$ is bounded with $M=\sup_k|\lambda_k|$, the estimate above gives
\[
\|Tx\|_2 \le M \|x\|_2,\qquad x\in\ell_2,
\]
so $T$ is bounded and $\|T\|\le M$. For the reverse inequality, let $e_k$ be the standard basis of $\ell_2$.
Then
\[
\|T e_k\|_2 = |\lambda_k|
\]
and $\|e_k\|_2=1$, so
\[
|\lambda_k| = \|T e_k\|_2 \le \|T\|\quad\text{for all }k.
\]
Taking the supremum in $k$ gives $\sup_k|\lambda_k|\le\|T\|$. Hence
\[
\|T\| = \sup_k |\lambda_k|,
\]
and in particular $T$ is bounded iff $(\lambda_k)$ is bounded. 
\medskip
(c) \emph{Compact.}
Suppose first that $\lambda_k\to 0$. For $N\in\mathbb{N}$ define the finite-rank operator
\[
T_N x := (\lambda_1 x_1,\dots,\lambda_N x_N,0,0,\dots).
\]
Then for $\|x\|_2\le 1$,
\[
\|(T-T_N)x\|_2^2 = \sum_{k>N} |\lambda_k x_k|^2
\le \big(\sup_{k>N} |\lambda_k|\big)^2 \sum_{k>N} |x_k|^2
\le \big(\sup_{k>N} |\lambda_k|\big)^2,
\]
so $\|T-T_N\|\le \sup_{k>N}|\lambda_k|\to 0$ as $N\to\infty$. Thus $T$ is a norm limit of finite-rank
operators, hence compact.

Conversely, assume $T$ is compact but $\lambda_k\not\to 0$. Then there exists $\varepsilon>0$ and a subsequence
$(\lambda_{k_j})$ with $|\lambda_{k_j}|\ge\varepsilon$ for all $j$. Consider the bounded sequence
$(e_{k_j})$ in $\ell_2$. Compactness of $T$ implies that $(Te_{k_j})$ has a convergent subsequence in $\ell_2$,
but
\[
Te_{k_j} = \lambda_{k_j} e_{k_j},\qquad \|Te_{k_j}\|_2 = |\lambda_{k_j}|\ge \varepsilon,
\]
and these vectors are pairwise orthogonal. An orthogonal sequence with norms bounded away from $0$ cannot have a
convergent subsequence, a contradiction. Hence we must have $\lambda_k\to 0$.


\end{proof}

    \item Consider an intergal operator $T$ with kernel $k(t,s) : [0,1]^2\rightarrow \mathbb{R}$ which satisfies the following:
    \begin{enumerate}
        \item for each $s\in [0,1],$ the function $k_s(t) = k(t,s)$ is intergrable in $t$;
        \item the map $s\rightarrow k_s$ is a continuous map from $[0,1] $ to $L_1[0,1].$
    \end{enumerate}
     Show that the integral operator $T$ is compact in $C[0,1].$

     \begin{proof}
First note $s\mapsto k_s\in L_1[0,1]$ is continuous on a compact set, so
\[
M:=\sup_{s\in[0,1]} \|k_s\|_{L_1}<\infty.
\]
Let $f\in C[0,1]$. Then for each $s\in[0,1]$,
\[
|(Tf)(s)| = \left|\int_0^1 k(t,s)f(t)\,dt\right|
\le \|f\|_\infty \int_0^1 |k(t,s)|\,dt
= \|f\|_\infty \|k_s\|_{L_1}
\le M\|f\|_\infty.
\]
Hence $Tf$ is bounded and $\|T\|\le M$.

We also get continuity: if $s_n\to s$, then
\[
|(Tf)(s_n)-(Tf)(s)|
= \left|\int_0^1 \big(k(t,s_n)-k(t,s)\big) f(t)\,dt\right|
\le \|f\|_\infty\,\|k_{s_n}-k_s\|_{L_1}\to 0,
\]
since $s\mapsto k_s$ is continuous in $L_1$. Thus $Tf\in C[0,1]$ and $T:C[0,1]\to C[0,1]$ is bounded.

To prove compactness, it suffices to show $T$ maps the unit ball
\[
B:=\{f\in C[0,1]:\|f\|_\infty\le 1\}
\]
into a precompact subset of $C[0,1]$. By Arzelà–Ascoli, it is enough to prove that $\{Tf:f\in B\}$ is uniformly bounded and equicontinuous.

Uniform boundedness: for $f\in B$ and $s\in[0,1]$,
\[
|(Tf)(s)| \le \|f\|_\infty \|k_s\|_{L_1} \le \|k_s\|_{L_1} \le M,
\]
so $\sup_{f\in B}\|Tf\|_\infty\le M$.

Equicontinuity: let $f\in B$ and $s,s_0\in[0,1]$. Then
\[
|(Tf)(s)-(Tf)(s_0)|
= \left|\int_0^1 \big(k(t,s)-k(t,s_0)\big)f(t)\,dt\right|
\le \|f\|_\infty\,\|k_s-k_{s_0}\|_{L_1}
\le \|k_s-k_{s_0}\|_{L_1}.
\]
Since $s\mapsto k_s$ is continuous into $L_1$ and $[0,1]$ is compact, this map is uniformly continuous. Thus for every $\varepsilon>0$ there exists $\delta>0$ such that
\[
|s-s_0|<\delta \implies \|k_s-k_{s_0}\|_{L_1}<\varepsilon,
\]
and hence
\[
|(Tf)(s)-(Tf)(s_0)|<\varepsilon
\]
for all $f\in B$ whenever $|s-s_0|<\delta$. This shows $\{Tf:f\in B\}$ is equicontinuous.

By Arzelà–Ascoli, the closure of $T(B)$ is compact in $C[0,1]$, so $T$ is compact.
\end{proof}

    \item Let $X$ be a Banach space and $T \in K(X,X)$. Show that operator $A=I-T$ satisfies
    $$
    dim(ker(A)) = dim(ker(A^*)) = codim(Im(A)) = codim(Im(A^*)).    
    $$

\begin{proof}
We work in several steps.

\medskip\noindent
\textbf{1. $\ker(A)$ and $\ker(A^*)$ are finite dimensional.}

Note that
\[
Ax=0 \iff x-Tx=0 \iff Tx=x,
\]
so $\ker(A)$ is the eigenspace of $T$ for eigenvalue $1$.

If $\ker(A)$ were infinite dimensional, we could choose a sequence
$(x_n)\subset\ker(A)$ with $\|x_n\|=1$ and no convergent subsequence (possible
because the unit ball of an infinite dimensional Banach space is not
compact). But then
\[
T x_n = x_n\quad\text{for all }n,
\]
so $(Tx_n)$ has no convergent subsequence either. This contradicts the
compactness of $T$. Hence $\ker(A)$ is finite dimensional.

The adjoint $T^*:X^*\to X^*$ is also compact (image of the unit ball under
$T^*$ is norm-compact in $X^*$ because $T$ is compact). Applying the same
argument to $A^*=I-T^*$ on $X^*$ shows $\ker(A^*)$ is finite dimensional.

\medskip\noindent
\textbf{2. $\operatorname{Im}(A)$ and $\operatorname{Im}(A^*)$ are closed.}

Let $N:=\ker(A)$, which is finite dimensional. There exists a closed
complement $M\subset X$ with
\[
X = N \oplus M.
\]
On $M$ the restriction $A|_M : M\to X$ is injective. We claim there is
$c>0$ such that
\begin{equation}\label{eq:lowerbound}
    \|Ax\|\ge c\|x\|\qquad\forall x\in M.
\end{equation}

If not, we can find $(x_n)\subset M$ with $\|x_n\|=1$ and $\|Ax_n\|\to 0$.
Then
\[
Ax_n = x_n - T x_n \to 0
\quad\Rightarrow\quad
x_n - Tx_n \to 0.
\]
Since $(x_n)$ is bounded and $T$ is compact, $(Tx_n)$ has a convergent
subsequence $Tx_{n_k}\to y$. Then
\[
x_{n_k} = Ax_{n_k} + Tx_{n_k} \to 0 + y = y,
\]
so $x_{n_k}\to y\in M$. Passing to the limit in $Ax_{n_k}\to 0$ gives
$Ay=0$, i.e.\ $y\in N\cap M = \{0\}$. Thus $y=0$, but $\|x_{n_k}\|=1$
forces $\|y\|=1$, a contradiction. Hence \eqref{eq:lowerbound} holds.

Now let $(Ax_n)$ be a convergent sequence in $A(M)$. Then for $m,n$,
\[
\|x_n - x_m\|
\le \frac{1}{c}\|A x_n - A x_m\|,
\]
so $(x_n)$ is Cauchy in $M$ and converges to some $x\in M$. Continuity of
$A$ gives $Ax_n\to Ax$, so $\lim Ax_n \in A(M)$. Thus $A(M)$ is closed.

Since $X=N\oplus M$ and $A(N)=\{0\}$, we have
\[
\operatorname{Im}(A)=A(X)=A(M).
\]
Therefore $\operatorname{Im}(A)$ is closed.

The same argument, applied to $A^*=I-T^*$ on the Banach space $X^*$, shows
that $\operatorname{Im}(A^*)$ is closed.

\medskip\noindent
\textbf{3. Relating codimension and adjoint kernels.}

We use a standard duality fact: if $Y$ is Banach and $M\subset Y$ is closed,
then
\[
(Y/M)^*\cong M^\perp := \{f\in Y^*: f|_M=0\}.
\]

Apply this with $Y=X$, $M=\operatorname{Im}(A)$. Then
\[
(X/\operatorname{Im}(A))^*
\cong \operatorname{Im}(A)^\perp.
\]
But
\[
\operatorname{Im}(A)^\perp
= \{f\in X^*: f(Ax)=0\ \forall x\}
= \ker(A^*).
\]
Hence
\[
(X/\operatorname{Im}(A))^* \cong \ker(A^*).
\]
So $(X/\operatorname{Im}(A))^*$ is finite dimensional, and therefore
$X/\operatorname{Im}(A)$ is finite dimensional with
\[
\dim(X/\operatorname{Im}(A))
= \dim\big((X/\operatorname{Im}(A))^*\big)
= \dim\ker(A^*).
\]
That is,
\begin{equation}\label{eq:codimA}
\operatorname{codim}\operatorname{Im}(A)
= \dim\ker(A^*).
\end{equation}

Now apply the same reasoning to $Y=X^*$ and $M=\operatorname{Im}(A^*)$. We
get
\[
(X^*/\operatorname{Im}(A^*))^*
\cong \operatorname{Im}(A^*)^\perp
= \ker\big((A^*)^*\big)
= \ker(A^{**}).
\]
The canonical embedding $J:X\to X^{**}$ satisfies $A^{**}J=J A$, so
$J(\ker(A))\subset\ker(A^{**})$, and conversely if $A^{**}u=0$ with
$u=Jx$, then $J(Ax)=0$, hence $Ax=0$ and $x\in\ker(A)$. Thus
$\ker(A^{**})\cong\ker(A)$ and
\[
\dim\ker(A^{**}) = \dim\ker(A).
\]
Therefore
\begin{equation}\label{eq:codimAstar}
\operatorname{codim}\operatorname{Im}(A^*)
= \dim\ker(A^{**})
= \dim\ker(A).
\end{equation}

So far we have
\[
\operatorname{codim}\operatorname{Im}(A) = \dim\ker(A^*),
\qquad
\operatorname{codim}\operatorname{Im}(A^*) = \dim\ker(A).
\]

\medskip\noindent
\textbf{4. Equality of $\dim\ker(A)$ and $\dim\ker(A^*)$.}

We have
\[
\ker(A) = \ker(I-T),\qquad
\ker(A^*) = \ker(I-T^*),
\]
which are the eigenspaces of $T$ and $T^*$ for the (nonzero) eigenvalue
$\lambda=1$.

For compact operators it is a standard fact that for every nonzero
$\lambda$, the eigenspaces $\ker(\lambda I - T)$ and $\ker(\lambda I - T^*)$
have the same (finite) dimension.

Applying this with $\lambda=1$ gives
\[
\dim\ker(A) = \dim\ker(A^*).
\]

\medskip

Combining this with \eqref{eq:codimA} and \eqref{eq:codimAstar}, we obtain
\[
\dim\ker(A)
= \dim\ker(A^*)
= \operatorname{codim}\operatorname{Im}(A)
= \operatorname{codim}\operatorname{Im}(A^*),
\]
as required.
\end{proof}








    
    \item Prove the claims about the spectra of shirt operators made in Example 4.3.8
    \[
\sigma_p(R)=\varnothing,\quad
\sigma_c(R)=\{\lambda:|\lambda|=1\},\quad
\sigma_r(R)=\{\lambda:|\lambda|<1\};
\]
\[
\sigma_p(L)=\{\lambda:|\lambda|<1\},\quad
\sigma_c(L)=\{\lambda:|\lambda|=1\},\quad
\sigma_r(L)=\varnothing.
\]

\begin{proof}
Let $(e_n)_{n\ge1}$ be the standard orthonormal basis of $\ell_2$.

\medskip\noindent
\textbf{1. Basic facts.}
For $x\in\ell_2$,
\[
\|Rx\|_2^2=\sum_{n\ge1}|(Rx)_n|^2=\sum_{n\ge1}|x_n|^2=\|x\|_2^2,
\]
so $R$ is an isometry and $\|R\|=1$. Similarly
\[
\|Lx\|_2^2=\sum_{n\ge1}|x_{n+1}|^2\le \sum_{n\ge1}|x_n|^2=\|x\|_2^2,
\]
and taking $x_1=0$ shows $\|L\|=1$. Hence for both operators the spectral
radius satisfies $r(\cdot)\le 1$, so
\[
\sigma(R),\sigma(L)\subset\{\lambda:|\lambda|\le 1\}.
\]

For $|\lambda|>1$ we have $\|R/\lambda\|<1$ and $\|L/\lambda\|<1$, so
\[
\lambda I - R = \lambda\Big(I-\tfrac1\lambda R\Big),\quad
\lambda I - L = \lambda\Big(I-\tfrac1\lambda L\Big)
\]
are invertible with inverses given by the Neumann series
$\sum_{n\ge0}(\tfrac1\lambda R)^n$ and $\sum_{n\ge0}(\tfrac1\lambda L)^n$.
Thus $|\lambda|>1$ lies in the resolvent set of both $R$ and $L$, and so
\[
\sigma(R),\sigma(L)\subset\{\lambda:|\lambda|\le 1\}.
\]

It is easy to check that $R^*=L$ and $L^*=R$, so $\sigma(R)=\sigma(L)$.

\medskip\noindent
\textbf{2. Eigenvalues of $R$ and $L$ (point spectrum).}

\emph{Right shift.} If $Rx=\lambda x$, then
\[
(0,x_1,x_2,\dots)=(\lambda x_1,\lambda x_2,\dots).
\]
From the first coordinate $\lambda x_1=0$, so for any $\lambda$ we get
$x_1=0$. Then $x_1=\lambda x_2$ gives $x_2=0$, and inductively $x_n=0$ for
all $n$. Hence $\ker(R-\lambda I)=\{0\}$ for every $\lambda\in\mathbb C$, so
\[
\sigma_p(R)=\varnothing.
\]

\emph{Left shift.} If $Lx=\lambda x$, we have
\[
(x_2,x_3,\dots)=(\lambda x_1,\lambda x_2,\dots),
\]
so $x_{n+1}=\lambda x_n$ for all $n\ge1$. Thus
\[
x_n=\lambda^{n-1}x_1,\qquad n\ge1.
\]
Then
\[
\|x\|_2^2 = |x_1|^2\sum_{n\ge1}|\lambda|^{2(n-1)}.
\]
This is finite iff $|\lambda|<1$. So for $|\lambda|<1$ we obtain a
one-dimensional eigenspace spanned by
\[
v^{(\lambda)}=(1,\lambda,\lambda^2,\dots),
\]
while for $|\lambda|\ge1$ the only solution is $x=0$. Therefore
\[
\sigma_p(L)=\{\lambda\in\mathbb C:|\lambda|<1\}.
\]

\medskip\noindent
\textbf{3. $0$ in the residual spectrum of $R$.}

$\operatorname{Im}R$ consists of all sequences whose first coordinate is $0$.
This is a closed subspace of codimension $1$, hence not dense in $\ell_2$.
Since $R$ is injective, $0$ belongs to the residual spectrum of $R$.

\medskip\noindent
\textbf{4. Residual / continuous spectrum via the adjoint.}

For any bounded operator $T$ on a Hilbert space and any $\lambda\in\mathbb C$,
\[
\overline{\operatorname{Im}(T-\lambda I)}^\perp
= \ker\big((T-\lambda I)^*\big)
= \ker(T^*-\overline{\lambda} I).
\]
In particular, if $T-\lambda I$ is injective, then
\[
\lambda\in\sigma_r(T)
\iff \overline{\operatorname{Im}(T-\lambda I)}\ne H
\iff \ker(T^*-\overline{\lambda}I)\ne\{0\}.
\]

We apply this with $T=R$ and $T=L$, using $R^*=L$, $L^*=R$.

\medskip\noindent
\textbf{5. Spectra for $R$.}

We already know $R-\lambda I$ is injective for all $\lambda$ (no eigenvalues).
For $|\lambda|<1$ we have $\overline{\lambda}\in\sigma_p(L)$ by Step~2, so
\[
\ker(L-\overline{\lambda} I)\ne\{0\}.
\]
Hence
\[
\overline{\operatorname{Im}(R-\lambda I)}^\perp
= \ker(L-\overline{\lambda}I)\ne\{0\},
\]
and $\operatorname{Im}(R-\lambda I)$ is not dense. Thus for $|\lambda|<1$,
$\lambda$ lies in the residual spectrum of $R$:
\[
\sigma_r(R)=\{\lambda:|\lambda|<1\}.
\]

For $|\lambda|=1$, we still have $\ker(R-\lambda I)=\{0\}$, but now
$\overline{\lambda}$ is not in $\sigma_p(L)$, so
\[
\ker(L-\overline{\lambda}I)=\{0\}
\quad\Rightarrow\quad
\overline{\operatorname{Im}(R-\lambda I)}^\perp=\{0\},
\]
so $\operatorname{Im}(R-\lambda I)$ is dense. On the other hand, for
$|\lambda|=1$ the vector $e_1$ is not in the range of $R-\lambda I$:
if $(R-\lambda I)x=e_1$, then coordinatewise
\[
-\lambda x_1 = 1,\quad x_1-\lambda x_2=0,\quad x_2-\lambda x_3=0,\dots
\]
which forces $x_n=-\lambda^{-n}$ and thus $x\notin\ell_2$. So
$R-\lambda I$ is not surjective. Therefore
\[
|\lambda|=1 \implies \lambda\in\sigma_c(R).
\]

Combining with $\sigma_p(R)=\varnothing$ and $\sigma(R)\subset\{|\lambda|\le1\}$,
we obtain
\[
\sigma_p(R)=\varnothing,\quad
\sigma_c(R)=\{\lambda:|\lambda|=1\},\quad
\sigma_r(R)=\{\lambda:|\lambda|<1\}.
\]

\medskip\noindent
\textbf{6. Spectra for $L$.}

We already have $\sigma_p(L)=\{\lambda:|\lambda|<1\}$. For residual spectrum,
note that if $\lambda\in\sigma_r(L)$ then $L-\lambda I$ is injective but
$\operatorname{Im}(L-\lambda I)$ is not dense, so
\[
\ker\big((L-\lambda I)^*\big)
= \ker(R-\overline{\lambda}I)\ne\{0\},
\]
which would make $\overline{\lambda}$ an eigenvalue of $R$. But $R$ has no
eigenvalues, so this is impossible; hence
\[
\sigma_r(L)=\varnothing.
\]

It remains to analyze $|\lambda|=1$. First, for $|\lambda|=1$ we saw in
Step~2 that $\ker(L-\lambda I)=\{0\}$ (no eigenvectors). We now show
$\lambda\in\sigma(L)$ by constructing approximate eigenvectors.

Fix $\lambda$ with $|\lambda|=1$. For each $N\in\mathbb N$ define
$x^{(N)}\in\ell_2$ by
\[
x^{(N)}_k=
\begin{cases}
\lambda^{k-1}, & 1\le k\le N,\\
0, & k>N.
\end{cases}
\]
Then
\[
(Lx^{(N)})_k = x^{(N)}_{k+1} =
\begin{cases}
\lambda^k, & 1\le k\le N-1,\\
0, & k\ge N,
\end{cases}
\]
so
\[
\big((L-\lambda I)x^{(N)}\big)_k=
\begin{cases}
0, & 1\le k\le N-1,\\
-\lambda^N, & k=N,\\
0, & k>N.
\end{cases}
\]
Hence $\|(L-\lambda I)x^{(N)}\|_2 = 1$. Meanwhile
\[
\|x^{(N)}\|_2^2 = \sum_{k=1}^N|\lambda|^{2(k-1)} = N,
\]
so for the unit vectors $u^{(N)}:=x^{(N)}/\sqrt{N}$ we get
\[
\|u^{(N)}\|_2=1,\qquad
\|(L-\lambda I)u^{(N)}\|_2 = \frac{1}{\sqrt{N}}\to 0.
\]
If $\lambda$ were in the resolvent of $L$, then $(L-\lambda I)^{-1}$ would be
bounded, say $\|(L-\lambda I)^{-1}\|\le C$, which would imply
\[
1=\|u^{(N)}\|
\le C\,\|(L-\lambda I)u^{(N)}\|\xrightarrow[N\to\infty]{}0,
\]
a contradiction. Thus $|\lambda|=1\Rightarrow \lambda\in\sigma(L)$.

Finally, for such $\lambda$,
\[
\overline{\operatorname{Im}(L-\lambda I)}^\perp
= \ker((L-\lambda I)^*)
= \ker(R-\overline{\lambda}I)
= \{0\},
\]
since $R$ has no eigenvalues. So $\operatorname{Im}(L-\lambda I)$ is dense.
We already know $L-\lambda I$ is not invertible (since $\lambda\in\sigma(L)$)
and has trivial kernel, so it cannot be surjective. Therefore
\[
|\lambda|=1 \implies \lambda\in\sigma_c(L).
\]

\end{proof}
    


\end{enumerate}

\end{document}
