\documentclass[]{article}

% Packages for mathematics
\usepackage{amsmath, amssymb, amsthm}
\usepackage{mathtools}
\usepackage{gensymb}
\usepackage{bm}
\usepackage{authblk}

% Package for graphics
\usepackage{graphicx}
\graphicspath{{../figures/}}
\usepackage{subcaption}
\usepackage{placeins}


% Package for page layout and headers/footers
%\usepackage{geometry}
%\geometry{margin=1in}


% Package for clickable links
\usepackage{hyperref}
\hypersetup{
    colorlinks=true,
    linkcolor=blue,
    citecolor=blue,
    urlcolor=blue,
}

% Package for algorithms
\usepackage{float}
\usepackage{algorithm}
\usepackage{algpseudocode}
\usepackage{matlab-prettifier}
\usepackage{tabularx}
\usepackage{amsmath, physics}

% Custom theorem environments
\newtheorem{theorem}{Theorem}[section]
\newtheorem{lemma}[theorem]{Lemma}
\newtheorem{corollary}[theorem]{Corollary}
\newtheorem{proposition}{Proposition}
\newtheorem{remark}{Remark}
\theoremstyle{definition}
\newtheorem{definition}[theorem]{Definition}

\title{Functional Analysis Homework 6}
\author{Kevin Ho}
\date{October 3rd}

\begin{document}

\maketitle

\begin{enumerate}
    \item Prove that $c^*_0 = \ell_1. $ The meaning of this is the same as in Corollary 2.2.6, i.e. the functionals on $c_0$ are given by summation with weight from $\ell_1$

\begin{proof}
Let $y=(y_k)\in \ell_1$ and define $T(y)\in (c_0)^*$ by
\[
(T(y))(x)=\sum_{k=1}^\infty x_k y_k,\qquad x=(x_k)\in c_0.
\]
This is well defined since
\[
\big|(T(y))(x)\big|\le \sum_{k=1}^\infty |x_k||y_k|
\le \|x\|_\infty \sum_{k=1}^\infty |y_k|=\|x\|_\infty\|y\|_1.
\]
Hence $T:\ell_1\to (c_0)^*$ is linear and bounded with $\|T(y)\|\le \|y\|_1$.

We claim $\|T(y)\|=\|y\|_1$. For $N\in\mathbb{N}$ set
$x^{(N)}=(x^{(N)}_k)$ by $x^{(N)}_k=\operatorname{sgn}(y_k)$ for $k\le N$ and
$0$ otherwise. Then $x^{(N)}\in c_0$, $\|x^{(N)}\|_\infty=1$, and
\[
(T(y))(x^{(N)})=\sum_{k=1}^N |y_k|.
\]
Therefore
\[
\|T(y)\|\ge \sup_{N}\big|(T(y))(x^{(N)})\big|=\sup_{N}\sum_{k=1}^N|y_k|
=\|y\|_1.
\]
Combined with $\|T(y)\|\le \|y\|_1$, this gives $\|T(y)\|=\|y\|_1$, so $T$ is an isometry.

It remains to show $T$ is surjective. Let $f\in (c_0)^*$. Define $y_k:=f(e_k)$, where
$e_k$ is the canonical basis vector. For $N\in\mathbb{N}$, set
$x^{(N)}=\sum_{k=1}^N \operatorname{sgn}(y_k)e_k\in c_0$. Then
\[
\sum_{k=1}^N |y_k|=\sum_{k=1}^N \operatorname{sgn}(y_k)f(e_k)=f(x^{(N)})
\le \|f\|\,\|x^{(N)}\|_\infty=\|f\|.
\]
Thus the partial sums are uniformly bounded, so $\sum_{k=1}^\infty |y_k|<\infty$ and
$y\in \ell_1$.

For $x\in c_{00}$ (finite support), by linearity,
\[
f(x)=\sum_{k=1}^\infty x_k f(e_k)=\sum_{k=1}^\infty x_k y_k=(T(y))(x).
\]
Since $c_{00}$ is dense in $c_0$ under $\|\cdot\|_\infty$ and both $f$ and $T(y)$ are continuous on $(c_0,\|\cdot\|_\infty)$, it follows that $f(x)=(T(y))(x)$ for all $x\in c_0$. Hence $f=T(y)$ with $y\in\ell_1$, proving $T$ is surjective.

Therefore $T:\ell_1\to (c_0)^*$ is a surjective isometry, i.e.\ $(c_0)^*=\ell_1$.
\end{proof}



\end{enumerate}

\end{document}
