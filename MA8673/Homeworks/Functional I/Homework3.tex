\documentclass[]{article}

% Packages for mathematics
\usepackage{amsmath, amssymb, amsthm}
\usepackage{mathtools}
\usepackage{gensymb}
\usepackage{bm}
\usepackage{authblk}

% Package for graphics
\usepackage{graphicx}
\graphicspath{{../figures/}}
\usepackage{subcaption}
\usepackage{placeins}


% Package for page layout and headers/footers
%\usepackage{geometry}
%\geometry{margin=1in}


% Package for clickable links
\usepackage{hyperref}
\hypersetup{
    colorlinks=true,
    linkcolor=blue,
    citecolor=blue,
    urlcolor=blue,
}

% Package for algorithms
\usepackage{float}
\usepackage{algorithm}
\usepackage{algpseudocode}
\usepackage{matlab-prettifier}
\usepackage{tabularx}

% Custom theorem environments
\newtheorem{theorem}{Theorem}[section]
\newtheorem{lemma}[theorem]{Lemma}
\newtheorem{corollary}[theorem]{Corollary}
\newtheorem{proposition}{Proposition}
\newtheorem{remark}{Remark}
\theoremstyle{definition}
\newtheorem{definition}[theorem]{Definition}

\title{Functional Analysis Homework 3}
\author{Kevin Ho}
\date{September 12th}

\begin{document}

\maketitle

\begin{enumerate}
    \item Let $X$ be a Banach space and $Y$ be a (linear) subspace of $X$. Show that :
    \begin{enumerate}
        \item $Y$ is a Banach space iff $Y$ is closed.
        \item If $Y$ is closed, then $X/Y$ is a Banach space.
    \end{enumerate}
    \item[Proof:]  By definition of complete we have that there should be the closed section of it. 
    
    (a) $(\Rightarrow)$ So let Y be a Banach Space. let $\{y_n\} \in Y$ such that $y_n \rightarrow x$ for $x\in X$. Note that Y is complete. With this we know $y_n$ is a Cauchy sequence and with the completeness property that $Y$ has, each $\{y_n\} \rightarrow y \in Y$. Since $\{y_n\}$ converges to both $x\in X $ and $y\in Y$, it must be that $x=y$ is true as limits in normed spaces are unique and so $x,y \in Y$ so Y is closed.

    (a) $(\Leftarrow)$ So now lets assume that Y is a closed subspace within X. Let $\{y_n\} \in Y$ such that $\{y_n\} $ is a Cauchy sequence. Since $Y\subseteq X$, $\{y_n\}$ is a Cauchy sequence in $X$. With this we can use the completeness property of X so that every sequence $\{y_n\}$ converges to some $x \in X$. By definition of $Y$ being closed, $x\in Y$. Thus we have shown the compleness of $Y$, meaning that Y is a Banach space.

    (b) So with this, we mainly need to show that $X/Y$ is complete. So let's define a Cauchy sequence $\{[x_n]\} \in X/Y$. We want to show that each sequence $\{[x_n]\}$ converges to some limit point in the space $X/Y$. Since $\{[x_n]\}$ is a Cauchy sequence, we can pick a subsequence $\{[x_{n_k}]\}$so that the distance between consecutive terms is less than $1/2^k$.
    So, for each $k \ge 1$, we have:
    $$\|[x_{n_{k+1}}] - [x_{n_k}]\|_{X/Y} = \|[x_{n_{k+1}} - x_{n_k}]\|_{X/Y} < \frac{1}{2^k}$$
     Using the definition of the quotient norm, for each $k$, we can find an element $y_k \in Y$ such that:
    $$\|(x_{n_{k+1}} - x_{n_k}) + y_k\|_X < \frac{1}{2^k}$$
    Let $v_k = x_{n_{k+1}} - x_{n_k} + y_k$. The series $\sum_{k=1}^{\infty} v_k$ is absolutely convergent because $\sum \|v_k\|_X < \sum \frac{1}{2^k} = 1$.Since $X$ is a Banach space, any absolutely convergent series converges. So, the sequence of partial sums $s_m = \sum_{k=1}^{m} v_k$ converges to some limit $s \in X$. Let's look at the partial sums again:
    $$s_m = \sum_{k=1}^{m} (x_{n_{k+1}} - x_{n_k} + y_k) = (x_{n_{m+1}} - x_{n_1}) + \sum_{k=1}^{m} y_k$$
    In the quotient space, the term $\sum y_k$ is in $Y$, so it vanishes. Thus, $[s_m] = [x_{n_{m+1}} - x_{n_1}]$. As $m \to \infty$, we have $s_m \to s$, so $[s_m] \to [s]$.
    This means $[x_{n_{m+1}} - x_{n_1}] \to [s]$, which implies that $[x_{n_{m+1}}] \to [s + x_{n_1}]$. Overall showing if a Cauchy sequence has a convergent subsequence, the entire sequence converges to the same limit. Therefore, $\{[x_n]\}$ converges in $X/Y$, proving that $X/Y$ is a Banach space
   \qed 
    



    
    \item Show that the space $C^k[0,1]$ of $k$-times differentiable functions is not a Banach space with respect to the sup-norm.

    Show that $C^k[0,1]$ is a Banach space with respect to the norm
    % So the main idea that comes with this is stuff that does not converge but continouts from 0-1 such as 1/x
    \begin{equation*}
        \|f\|_{C^k} = \|f\|_{\infty} + \|f'\|_{\infty} + ... +\|f^{(k)}\|_{\infty}. 
    \end{equation*}

    \item[Proof:] So let's observe the sequence of functions:
    \begin{equation*}
        f_n(x) = \sqrt{({x-\frac{1}{2}})^2 + \frac{1}{n}}
    \end{equation*}
    so let's observe as $n\rightarrow\infty$. The $\frac{1}{n}$ goes to zero and so know that the function is pointwise continuous and since the function under the square root is always positive for the interval $[0,1]$, we know that it is k-times differentiable. So now let's show that it is not a Banach space w.r.t the sup-norm. 
   
    So if we observe the derivative of the function at the point $\frac{1}{2}$, there is a discontinuity at that point(sharp corner at that x value). thus is not complete therefore the space $C^k[0,1]$ of $k$-times differentiable functions is not a Banach space with respect to the sup-norm.

    So now with that in mind, let's show that the space is a Banach space with the given norm $\|f\|_{C^k}$. So let $\{f_n\} \in (C^k[0,1],\|\cdot\|_{C^k})$ be Cauchy. This gives us that for every $\epsilon > 0$ 
    \begin{equation*}
        \|f_n-f_m\|_{C^k}= \sum_{j=0}^k\\||f_n^{(j)}-f_m^{(j)}\|_\infty < \epsilon \Rightarrow \||f_n^{(j)}-f_m^{(j)}\|_\infty < \epsilon
    \end{equation*}
    So let's pull a result that we have done in Adv Calc II think where we know that the space $(C[0,1], \|\cdot\|_\infty)$ is a Banach Space and so complete. With this combining from the earlier inequality, there is a limit point in each continuous function let's call $g_j$, i.e. $f_n^{(j)} \rightarrow g_j$. Now, let $f=g_0$.

    \begin{equation*}
        f'_n(x) - f'_n(0) = \int_0^x f''(t)dt
    \end{equation*}
    \begin{equation*}
        \Rightarrow \lim_{n\rightarrow \infty}(f'_n(x) - f'_n(0)) =  \lim_{n\rightarrow \infty}\int_0^x f''_n(t)dt
    \end{equation*}
    \begin{equation*}
    \Rightarrow g_1(x) - g_1(0) =  \int_0^x\lim_{n\rightarrow \infty} f''_n(t)dt
    \end{equation*}
    \begin{equation*}
    \Rightarrow g_1(x) - g_1(0) =  \int_0^xg_2(t)dt
    \end{equation*}

    And so essentially we get that $g_n' = g_{n+1}$ as we keep repeating this for k-times. And so by induction, we show that $f^(j) = g_j$ for all $j = 1,..,k$. Since each $g_j$ is continuous, this proves that f is k-times continuously differentiable, so $f \in C^k[0,1]$

    \begin{equation*}
        \|f_n-f\|_{C^k}= \sum_{j=0}^k\\||f_n^{(j)}-f^{(j)}\|_\infty  \Rightarrow \||f_n^{(j)}-f^{(j)}\|_\infty 
    \end{equation*}

     \begin{equation*}
        \Rightarrow \lim_{n\rightarrow \infty} \|f_n-f\|_{C^k} =  \sum_{j=0}^k (\lim_{n\rightarrow \infty}||f_n^{(j)}-g_j\|_\infty) = 0 
    \end{equation*}

    Therefore the space is complete and is a Banach space. \qed
    



    \item Prove the polarization identity.
    \begin{equation*}
        \langle x,y \rangle = \frac{1}{4}(\|x+y\|^2 - \|x-y\|^2 +i\|x+iy\|^2 - i\|x-iy\|^2)
    \end{equation*}
    %just look at previous theorems and properties that exist occur for inner product spaces.
    \item[Proof:] 
    \begin{equation*}
        \frac{1}{4}(\|x+y\|^2 - \|x-y\|^2 +i\|x+iy\|^2 - i\|x-iy\|^2)
    \end{equation*}
    \begin{equation*}
        \|x+y\|^2 \Rightarrow \|x\|^2 + 2Re \langle x,y \rangle + \|y\| ^2
    \end{equation*}
    \begin{equation*}
        \|x-y\|^2 \Rightarrow \|x\|^2 - 2Re \langle x,y \rangle + \|y\| ^2
    \end{equation*}
    combining them simplifies to:
    \begin{equation*}
        \frac{1}{4}(4Re\langle x,y \rangle + i\|x+iy\|^2 - i\|x-iy\|^2)
    \end{equation*}
    So now let's simplify the other two norms in the equation:
    \begin{equation*}
        \|x+iy\|^2 \Rightarrow \langle x+iy, x+iy\rangle  \Rightarrow 
        \langle x,x\rangle + \langle x,iy\rangle + \langle iy,x\rangle +\langle iy,iy\rangle
    \end{equation*}
    \begin{equation*}
        \Rightarrow\|x\|^2 - i\langle x,y\rangle +i\langle y,x\rangle  + i(-i)\|y\|^2 
    \end{equation*}
    \begin{equation*}
        \Rightarrow\|x\|^2 - i\langle x,y\rangle + i\overline{\langle x,y\rangle}  + \|y\|^2 
    \end{equation*}

    \begin{equation*}
        \|x-iy\|^2 \Rightarrow \langle x-iy, x-iy\rangle  \Rightarrow 
        \langle x,x\rangle - \langle x,iy\rangle - \langle iy,x\rangle +\langle iy,iy\rangle
    \end{equation*}
    \begin{equation*}
        \Rightarrow\|x\|^2 + i\langle x,y\rangle -i\langle y,x\rangle  + i(-i)\|y\|^2 
    \end{equation*}
    \begin{equation*}
        \Rightarrow\|x\|^2 + i\langle x,y\rangle -i\overline{\langle x,y\rangle}  + \|y\|^2 
    \end{equation*}

    And then combining them together gets 
    \begin{equation*}
        i(\|x+iy\|^2 -\|x-iy\|^2) \Rightarrow i((-2i)\langle x,y\rangle +(2i) 
        \overline{\langle x,y\rangle})
    \end{equation*}
    \begin{equation*}
        \Rightarrow (i(-2i)(\langle x,y\rangle - \overline{\langle x,y\rangle})
    \end{equation*}
    \begin{equation*}
        \Rightarrow (2)2iIm(\langle x,y\rangle))
    \end{equation*}
    \begin{equation*}
        \Rightarrow 4iIm(\langle x,y\rangle))
    \end{equation*}

    And then we can plug that into our earlier simplification to get
    \begin{equation*}
        \frac{1}{4}(4Re\langle x,y \rangle + 4iIm(\langle x,y\rangle)) \Rightarrow\langle x,y\rangle
    \end{equation*}

    \qed

    
    



    \item Let $X,Y $ be inner product spaces. Show that their direct sum 
    \begin{equation*}
        X \oplus_2 Y:= \{(x,y): x\in X, y\in Y\}
    \end{equation*}
    is also an inner product space, with the inner product defined as 
    \begin{equation*}
        \langle(x_1,y_1), (x_2,y_2)\rangle = \langle x_1,x_2 \rangle + \langle y_1,y_2 \rangle.
    \end{equation*}
        Derive a formula for the norm in $X \oplus_2 Y$. Show that if $X,Y$ are Hilbert spaces then so is $X \oplus_2 Y$.

    \item[Proof:] So let's show the three axioms for the inner product space
    \begin{enumerate}

        \item[Positive-Definiteness:] We check the inner product of a vector $u_1 = (x_1, y_1)$ with itself:
    \[ \langle u_1, u_1 \rangle = \langle (x_1,y_1), (x_1,y_1) \rangle = \langle x_1, x_1 \rangle_X + \langle y_1, y_1 \rangle_Y \]
    Since $X$ and $Y$ are inner product spaces, $\langle x_1, x_1 \rangle_X \ge 0$ and $\langle y_1, y_1 \rangle_Y \ge 0$. Their sum is also non-negative, so $\langle u_1, u_1 \rangle \ge 0$.
    
    Furthermore, $\langle u_1, u_1 \rangle = 0$ if and only if $\langle x_1, x_1 \rangle_X + \langle y_1, y_1 \rangle_Y = 0$. Since both terms are non-negative, this holds if and only if $\langle x_1, x_1 \rangle_X = 0$ and $\langle y_1, y_1 \rangle_Y = 0$. By the axioms in $X$ and $Y$, this implies $x_1 = \mathbf{0}$ and $y_1 = \mathbf{0}$, so $u_1 = (\mathbf{0}, \mathbf{0})$.
    Furthermore, $\langle u_1, u_1 \rangle = 0$ if and only if $\langle x_1, x_1 \rangle_X + \langle y_1, y_1 \rangle_Y = 0$. Since both terms are non-negative, this holds if and only if $\langle x_1, x_1 \rangle_X = 0$ and $\langle y_1, y_1 \rangle_Y = 0$. By the axioms in $X$ and $Y$, this implies $x_1 = \mathbf{0}$ and $y_1 = \mathbf{0}$, so $u_1 = (\mathbf{0}, \mathbf{0})$.
        So with the fact that we are using the inner products and adding them together, we can assume positive definiteness as unique.
        \item[Linearity:]\begin{align*}
        \langle a u_1 + u_2, u_3 \rangle &= \langle (ax_1 + x_2, ay_1 + y_2), (x_3, y_3) \rangle \\
        &= \langle ax_1 + x_2, x_3 \rangle_X + \langle ay_1 + y_2, y_3 \rangle_Y \\
        &= (a\langle x_1, x_3 \rangle_X + \langle x_2, x_3 \rangle_X) + (a\langle y_1, y_3 \rangle_Y + \langle y_2, y_3 \rangle_Y)  \\
        &= a(\langle x_1, x_3 \rangle_X + \langle y_1, y_3 \rangle_Y) + (\langle x_2, x_3 \rangle_X + \langle y_2, y_3 \rangle_Y)  \\
        &= a \langle u_1, u_3 \rangle + \langle u_2, u_3 \rangle 
    \end{align*}
    \item[Conjugate Sym:]
    \begin{align*}
        \langle u_1, u_2 \rangle &= \langle x_1, x_2 \rangle_X + \langle y_1, y_2 \rangle_Y \\
        &= \overline{\langle x_2, x_1 \rangle_X} + \overline{\langle y_2, y_1 \rangle_Y} \\
        &= \overline{\langle x_2, x_1 \rangle_X + \langle y_2, y_1 \rangle_Y} \\
        &= \overline{\langle u_2, u_1 \rangle}
    \end{align*}
    \end{enumerate}
    The norm is defined as $\|u\| = \sqrt{\langle u, u \rangle}$. For a vector $u = (x,y)$:
\[ \langle u, u \rangle = \langle x, x \rangle_X + \langle y, y \rangle_Y = \|x\|_X^2 + \|y\|_Y^2 \]
Therefore, the formula for the norm is:
\[ \|(x,y)\| = \sqrt{\|x\|_X^2 + \|y\|_Y^2} \]




    Suppose $X$ and $Y$ are Hilbert spaces. We must show $X \oplus_2 Y$ is complete.

     Let $\{u_n\}$ be a Cauchy sequence in $X \oplus_2 Y$, where $u_n = (x_n, y_n)$. For any $\epsilon > 0$, there is an $N$ such that for all $m, n > N$:
    \[ \|u_n - u_m\|^2 = \|x_n - x_m\|_X^2 + \|y_n - y_m\|_Y^2 < \epsilon^2 \]
    
 Since both terms are non-negative, this implies $\|x_n - x_m\|_X^2 < \epsilon^2$ and $\|y_n - y_m\|_Y^2 < \epsilon^2$. Thus, $\{x_n\}$ is a Cauchy sequence in $X$ and $\{y_n\}$ is a Cauchy sequence in $Y$. Since $X$ and $Y$ are complete, these sequences converge to limits: $x_n \to x \in X$ and $y_n \to y \in Y$. Let $u = (x,y)$. We show that $u_n \to u$:
    \[ \lim_{n \to \infty} \|u_n - u\|^2 = \lim_{n \to \infty} (\|x_n - x\|_X^2 + \|y_n - y\|_Y^2)  = 0 \]

Since every Cauchy sequence in $X \oplus_2 Y$ converges to a limit in the space, it is complete. Therefore, if $X$ and $Y$ are Hilbert spaces, so is $X \oplus_2 Y$. \qed


    \item Show that $C(K), c_0, L_p[0,1], \ell _p$ for $p=[1, \infty]$, $p\neq 2$ are not inner product spaces. (More accurately, it is not possible to define an inner product on those spaces which would agree with their norms). Use Theorem 1.4.21.
    Note for myself: $\|x+y\|^2 + \|x-y\|^2 = 2(\|x\|^2 + \|y\|^2)$
    

    \begin{enumerate}
        \item [$C(K)$:] Consider the space $C[0,1]$ with the sup-norm. Let $f=x, g=1$
        \begin{enumerate}
            \item $\|f\|_\infty$ = $1$
            
            \item $\|g\|_\infty$ = $1$
            
            \item $\|f+g\|_\infty$ = $2$
            
            \item $\|f-g\|_\infty$ = $1$
        \end{enumerate}
        We then get $2^2+1^2 = 2(1^1+1^1)$ Which is not true so $C(K)$ is not an inner product space.

        \item[$c_0$]: let $x = (1,0,0,...), y = (0,1,0,...)$ with the sup norm
        \begin{enumerate}
            \item $\|x\|_\infty$ = $1$
            
            \item $\|y\|_\infty$ = $1$
            
            \item $\|x+y\|_\infty$ = $1$
            
            \item $\|x-y\|_\infty$ = $1$
        \end{enumerate}
        We then get $1^2+1^2 = 2(1^1+1^1)$ Which is not true so $c_0$ is not an inner product space.
        
       \item [$L_p$ for $p\neq2$]: let $f = \chi_{[0,\frac{1}{2}]}(t), g =\chi_{[\frac{1}{2},1]}(t) $ with the p norm $(\int_0^1|f(t)|^pdt)^\frac{1}{p}$
        \begin{enumerate}
            \item $\|x\|_p$ = $(\frac{1}{2})^\frac{1}{p}$
            
            \item $\|y\|_p$ = $(\frac{1}{2})^\frac{1}{p}$
            
            \item $\|x+y\|_p$ = $1$
            
            \item $\|x-y\|_p$ = $1$
        \end{enumerate}
        We then get $1^2+1^2 = 2((\frac{1}{2})^\frac{2}{p}+(\frac{1}{2})^\frac{2}{p}) \rightarrow 4((\frac{1}{2})^\frac{2}{p}) $ Which is not true unless p = 2 so $L_p$ for $p\neq 2$ is not an inner product space.


        \item [$\ell_p$ for $p\neq2$]: let $x = (1,0,0,...), y = (0,1,0,...)$  with the p norm $(\sum_{k=1}^\infty|x|^p)^\frac{1}{p}$
        \begin{enumerate}
            \item $\|x\|_p$ = $1$
            
            \item $\|y\|_p$ = $1$
            
            \item $\|x+y\|_p$ = $(2)^\frac{1}{p}$
            
            \item $\|x-y\|_p$ = $(2)^\frac{1}{p}$
        \end{enumerate}
        We then get $2^\frac{2}{p}+2^\frac{2}{p} = 2(1^1+1^1)$ Which is not true unless p = 2 so $\ell_p$ for $p\neq 2$ is not an inner product space.

        
        
    \end{enumerate}
    \qed
    


\end{enumerate}

\end{document}
