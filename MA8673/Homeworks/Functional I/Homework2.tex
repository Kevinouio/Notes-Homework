\documentclass[]{article}

% Packages for mathematics
\usepackage{amsmath, amssymb, amsthm}
\usepackage{mathtools}
\usepackage{gensymb}
\usepackage{bm}
\usepackage{authblk}

% Package for graphics
\usepackage{graphicx}
\graphicspath{{../figures/}}
\usepackage{subcaption}
\usepackage{placeins}


% Package for page layout and headers/footers
%\usepackage{geometry}
%\geometry{margin=1in}


% Package for clickable links
\usepackage{hyperref}
\hypersetup{
    colorlinks=true,
    linkcolor=blue,
    citecolor=blue,
    urlcolor=blue,
}

% Package for algorithms
\usepackage{float}
\usepackage{algorithm}
\usepackage{algpseudocode}
\usepackage{matlab-prettifier}
\usepackage{tabularx}

% Custom theorem environments
\newtheorem{theorem}{Theorem}[section]
\newtheorem{lemma}[theorem]{Lemma}
\newtheorem{corollary}[theorem]{Corollary}
\newtheorem{proposition}{Proposition}
\newtheorem{remark}{Remark}
\theoremstyle{definition}
\newtheorem{definition}[theorem]{Definition}

\title{Functional Analysis Homework 1}
\author{Kevin Ho}
\date{August 29th}

\begin{document}

\maketitle

\begin{enumerate}
    \item Many of linear vector spaces introduced in Section 1.1.2 and Example 1.1.12 are in fact normed spaces. Check the norm axioms for them:


    
    \begin{enumerate}
        \item The space of bounded sequences $\ell_\infty$ is a normed space, with the normed defined as 
        \begin{equation*}
            \left\lVert x \right\rVert _\infty := \sup_i|x_i|.
        \end{equation*}
        \item[Proof:] 
            So for the first property for Positive definiteness, since we take the absolute value of each $x_i$ term, it checks the Positive part. 
            
           Definiteness ($\implies$ direction): Assume $||x||_\infty = 0$. By definition of supremum, $\sup_i |x_i| = 0$. Since the supremum is the least upper bound, $0$ must be an upper bound for the set $\{|x_i|\}$. This means $|x_i| \le 0$ for all $i$. Combining this with $|x_i| \ge 0$, we conclude that $|x_i| = 0$ for all $i$. This implies $x_i = 0$ for all $i$, so $x = \mathbf{0}$
\newline

            Definiteness ($\impliedby$ direction): Assume $x = \mathbf{0}$. Then $x_i = 0$ for all $i$. The set of values $\{|x_i|\}$ is simply $\{0\}$. The supremum of $\{0\}$ is $0$. Therefore, $||x||_\infty = 0$.

            So for the homogeneity, be the properties of the norm being defined as the sup, the following happens.

            \begin{equation*}
                \|cx\|_\infty = \sup_i|cx_i| = |c|\sup_i|x_i| = |c|\|x\|_\infty
            \end{equation*}

            And finally for the Triangle Inequality axiom, we have the following:

            For any specific index $k$, we apply the standard triangle inequality for real numbers:
            $$|x_k + y_k| \le |x_k| + |y_k|$$
            By definition of the supremum norm, $|x_k| \le \sup_i |x_i| = ||x||_\infty$ and $|y_k| \le \sup_i |y_i| = ||y||_\infty$. Substituting these into the inequality above gives:
            $$|x_k + y_k| \le ||x||_\infty + ||y||_\infty$$
            This inequality holds true for every index $k$. The term on the right side, $||x||_\infty + ||y||_\infty$, is therefore an upper bound for the set $\{|x_k + y_k| \mid k \in \mathbb{N}\}$. Since the supremum ($||x+y||_\infty$) is the least upper bound, it must be less than or equal to any other upper bound.
            Therefore:
            $$||x + y||_\infty = \sup_k |x_k + y_k| \le ||x||_\infty + ||y||_\infty$$
    \qed




        
        \item The space $c$ and $c_0$ are normed spaces, with the same sup-norm as above
        \item[Proof:] So the norm axioms are as exactly as written for proving $\ell_\infty$ is a normed space so I am not going to rewrite them but instead I will show that $c $ and $c_0$ are a subspace of $\ell_\infty$ , we need to show that the zero vector exist in both, closed under addition and multiplication. 
        No problem. Your approach is exactly right. Since the norm axioms are inherited from $\ell_\infty$, we just need to confirm that $c$ and $c_0$ are valid vector subspaces.


\textbf{Proof for $c$:}
The zero sequence is $\mathbf{0} = (0, 0, 0, \dots)$. Since $\lim_{i \to \infty} 0 = 0$, the limit exists, which means $\mathbf{0} \in c$.

We check if the sum $x + y$ is in $c$. Using the properties of limits:
    $$\lim_{i \to \infty} (x_i + y_i) = \lim_{i \to \infty} x_i + \lim_{i \to \infty} y_i = L_1 + L_2$$
    Since $L_1 + L_2$ is a finite value, the limit exists. Therefore, $x + y \in c$.
    
    We check if $\alpha x$ is in $c$ for a scalar $\alpha$. Using the properties of limits:
    $$\lim_{i \to \infty} (\alpha x_i) = \alpha \left( \lim_{i \to \infty} x_i \right) = \alpha L_1$$
    Since $\alpha L_1$ is a finite value, the limit exists. Therefore, $\alpha x \in c$.


 \textbf{Proof for $c_0$:} We show that $c_0$ is a subspace of $\ell_\infty$. Let $x = (x_i)$ and $y = (y_i)$ be sequences in $c_0$. By definition, $\lim_{i \to \infty} x_i = 0$ and $\lim_{i \to \infty} y_i = 0$.

The zero sequence $\mathbf{0} = (0, 0, 0, \dots)$ converges to 0, so $\mathbf{0} \in c_0$.

We check the sum $x + y$:
    $$\lim_{i \to \infty} (x_i + y_i) = \lim_{i \to \infty} x_i + \lim_{i \to \infty} y_i = 0 + 0 = 0$$
    Since the limit is 0, $x + y \in c_0$.

We check $\alpha x$ for a scalar $\alpha$:
    $$\lim_{i \to \infty} (\alpha x_i) = \alpha \left( \lim_{i \to \infty} x_i \right) = \alpha \cdot 0 = 0$$
    Since the limit is 0, $\alpha x \in c_0$. \qed

        
        \item The space of summable sequences $\ell_1$ is a normed space, with the norm defined as 
        \begin{equation*}
            \left\lVert x \right\rVert _1 := \sum^\infty_{i=1} |x_i|.
        \end{equation*}
        \item[Proof:]  So the first axiom of positiveness is satisfied as we are taking the sum of absolute values of each term $x_i$ so thus always positive. For definiteness, let $x=0$. If that is the case then each term within the sum is 0 therefore $\|x\|_1 = 0$. And if $\|x\|_1 = 0$, this would imply that each $|x_i|$ term has to equal 0 so for $\|x\|_1 = 0$ so thus positive definiteness is satisfied.

        For homogeneity, the following occurs.
        \begin{equation*}
            \left\lVert cx \right\rVert _1 := \sum^\infty_{i=1} |cx_i| = |c|\sum^\infty_{i=1} |x_i| = |c|\|x\|_1
        \end{equation*}

        Now to show triangle inequality is straightforward. 

        \begin{equation*}
            \left\lVert x+y \right\rVert _1 := \sum^\infty_{i=1} |x_i + y_i| \leq \sum^\infty_{i=1} |x_i| +\sum^\infty_{i=1} |y_i| = \|x\|_1 + \|y\|_1
        \end{equation*}
\qed
        
        \item The space $C(K)$ of continuous functions on a compact topological space $K$ is a normed space with the norm 
        \begin{equation*}
           \|f\|_{\infty} := \max_K |f(x)|.
        \end{equation*}
        \item[Proof:]For any $x \in K$, $|f(x)| \ge 0$. The maximum of a set of non-negative numbers must also be non-negative, so $\|f\|_\infty \ge 0$. Assume $\|f\|_\infty = 0$. This means $\max_{x \in K} |f(x)| = 0$. Since $|f(x)| \ge 0$ for all $x$, and the maximum value is 0, we must have $|f(x)| = 0$ for every single point $x \in K$. This implies $f(x) = 0$ for all $x$, meaning $f$ is the zero function $\mathbf{0}$. Assume $f = \mathbf{0}$. Then $f(x) = 0$ for all $x \in K$. Therefore, $\|f\|_\infty = \max_{x \in K} |0| = 0$. Thus satisfying Positive Definiteness Axiom.
        
        So now lets check homogeneity. Let $\alpha$ be a scalar.
        \begin{equation*}
            \|\alpha f\|_{\infty} := \max_K |\alpha f(x)| = |\alpha|  \max_K |f(x)| = |\alpha|\|f\|_\infty
        \end{equation*}

        Now let's check the triangle inequality. let $g\in K$
        
        Let $f, g \in C(K)$. For any point $x \in K$, apply the standard triangle inequality for real numbers:
        $$|f(x) + g(x)| \le |f(x)| + |g(x)|$$
        By definition of the maximum norm, $|f(x)| \le \max_{y \in K} |f(y)| = \|f\|_\infty$ and $|g(x)| \le \max_{y \in K} |g(y)| = \|g\|_\infty$. Substitute these into the inequality:
        $$|f(x) + g(x)| \le \|f\|_\infty + \|g\|_\infty$$
        This inequality holds true for every $x \in K$. Therefore, the term $\|f\|_\infty + \|g\|_\infty$ is an upper bound for all values of $|f(x) + g(x)|$. The maximum value of $|f(x) + g(x)|$ cannot exceed this upper bound.
        $$\|f + g\|_\infty = \max_{x \in K} |f(x) + g(x)| \le \|f\|_\infty + \|g\|_\infty$$
        \qed
        
        \item The space $L_1 = L_1(\Omega, \Sigma, \mu)$ is a normed space, with the norm defined as 
        \begin{equation*}
           \|f\|_1 := \int_{\Omega} |f(x)| \, d\mu.
        \end{equation*}
        Note that $\ell_1$ us a partial case of the space $L_1(\Omega, \Sigma, \mu)$ where $\Omega = \mathbb{N}$ and $\mu$ is the counting measure on $\mathbb{N}$.



        \item[Proof:] So for the first axiom, positive definiteness, we first check non-negativity. Since the norm is defined as the integral of an absolute value, and $|f(x)| \ge 0$ for all $x$, the integral must be non-negative.

        Now for definiteness, first let $f = \mathbf{0}$ be the zero function. Then $\|f\|_1 = \int_{\Omega} |0| \, d\mu = 0$. For the other direction, let $\|f\|_1 = 0$. This means $\int_{\Omega} |f(x)| \, d\mu = 0$. A fundamental property of integration states that for a non-negative function $|f(x)|$, the integral is zero if and only if $f(x) = 0$ almost everywhere. In the context of $L_1$ spaces, functions that are equal almost everywhere are considered the same element. Thus, $f$ is equivalent to the zero function in this space. So positive definiteness is satisfied.

        For homogeneity we have the following. Let $\alpha$ be a scalar.

        \begin{equation*}
            \|cf\|_1 = \int_{\Omega} |cf(x)| \, d\mu= |c|\int_{\Omega} |f(x)| \, d\mu = |c|\|f\|_1
        \end{equation*}


        Let $f, g \in L_1$. We start with the pointwise triangle inequality for real numbers, which holds for every $x \in \Omega$:
        $$|f(x) + g(x)| \le |f(x)| + |g(x)|$$
        
        Next, we integrate both sides over $\Omega$. By the monotonicity property of the Lebesgue integral, we have:
        $$\int_{\Omega} |f(x) + g(x)| \, d\mu \le \int_{\Omega} (|f(x)| + |g(x)|) \, d\mu = \int_{\Omega} |f(x)| \, d\mu + \int_{\Omega} |g(x)| \, d\mu$$
        
        Combining these steps and substituting the norm definition gives:
        $$\|f + g\|_1 \le \|f\|_1 + \|g\|_1$$
        \qed



        
        \item The space $L_\infty = L_\infty(\Omega, \Sigma, \mu)$ is a normed space, with the norm defined the **essential supremum**
        \begin{equation*}
            \|f\|_{\infty} := \underset{t \in \Omega}{\operatorname{ess\,sup}} |f(t)| := \inf_{g=f \text{ a.e.}} \sup_{t \in \Omega} |g(t)|.
        \end{equation*}
        \item[Proof:]  So for the first axiom, positive definiteness, we first check non-negativity. The norm $\|f\|_\infty$ represents the essential supremum for $|f(x)|$. Since $|f(x)| \ge 0$ for all $x$, any upper bound for it must also be non-negative. Therefore it is non-negative. 
        
        Now for definiteness, first let $f = \mathbf{0}$. This means $f(x) = 0$ almost everywhere. The smallest value $C$ such that $|f(x)| \le C$ almost everywhere is $C=0$. Thus, $\|f\|_\infty = 0$.
        
        For the other direction, assume $\|f\|_\infty = 0$. This means the essential supremum of $|f(x)|$ is $0$. If the function $|f(x)|$ were greater than zero on a set of positive measure, the essential supremum would also have to be greater than zero. Therefore, we must have $f(x) = 0$ almost everywhere. As with $L_1$, functions that are equal almost everywhere are considered the same element in $L_\infty$, so $f$ is equivalent to the zero vector. Thus, positive definiteness is satisfied.
        




Let $C = \|f\|_\infty$ and $\alpha$ be a scalar. By definition, this means $|f(x)| \le C$ almost everywhere.
Let's multiply this inequality by $|\alpha|$:
$$|\alpha f(x)| = |\alpha| \cdot |f(x)| \le |\alpha| \cdot C$$
This shows that $|\alpha|C$ is an essential upper bound for the function $\alpha f$. Since the norm $\|\alpha f\|_\infty$ is the smallest possible essential upper bound, it must be less than or equal to the bound we just found:
$$\|\alpha f\|_\infty \le |\alpha|C = |\alpha|\|f\|_\infty$$

To show equality, we can apply this result again. If $\alpha \neq 0$:
$$\|f\|_\infty = \|\frac{1}{\alpha}(\alpha f)\|_\infty \le \left|\frac{1}{\alpha}\right| \|\alpha f\|_\infty = \frac{1}{|\alpha|} \|\alpha f\|_\infty$$
Rearranging this gives $|\alpha| \|f\|_\infty \le \|\alpha f\|_\infty$. Since we have proved both directions ($\le$ and $\ge$), homogeneity holds.


Finally, let's check the triangle inequality. 

Let $\|f\|_\infty = C_1$ and $\|g\|_\infty = C_2$.
From the definition of essential supremum, we know:
1.  $|f(x)| \le C_1$ almost everywhere.
2.  $|g(x)| \le C_2$ almost everywhere.

The set where condition 1 fails has measure zero, and the set where condition 2 fails has measure zero. The union of these two sets also has measure zero. Therefore, both conditions hold simultaneously almost everywhere.

Now, apply the standard triangle inequality for numbers at every point $x$ where both bounds hold:
$$|f(x) + g(x)| \le |f(x)| + |g(x)| \le C_1 + C_2$$

This shows that $C_1 + C_2$ serves as an essential upper bound for the function $|f + g|$. Since $\|f + g\|_\infty$ is the greatest lower bound of all possible essential upper bounds, it must be less than or equal to this specific upper bound.
$$\|f + g\|_\infty \le C_1 + C_2 = \|f\|_\infty + \|g\|_\infty$$
Thus, the triangle inequality holds. 
\qed
        
    \end{enumerate}
        

    

    \item Prove that the norm assignment $x \mapsto \|x\|$ is a continuous function on the normed space. Specifically, show that if $\|x_n - x\| \rightarrow0$ then $\|x_n\| \rightarrow \|x\|$.
    \item[Proof:] So let's start with the inequality below as we know the inequality to be true as due to the triangle inequality.
    
    $$0 \le |\|x_n\| - \|x\|| \le \|x_n - x\|$$

    We are given that $\|x_n - x\| \to 0$ as $n \to \infty$. By the Squeeze Theorem, since $|\|x_n\| - \|x\||$ is trapped between $0$ and a sequence converging to $0$, it must also converge to $0$.
    $$\lim_{n \to \infty} |\|x_n\| - \|x\|| = 0$$
    This is precisely the definition of convergence for real numbers, so $\lim_{n \to \infty} \|x_n\| = \|x\|$. \qed
    


    \item Let $X$ and $Y$ be two normed spaces. Consider their direct (Cartesion) product 
    \begin{equation*}
        X \oplus_1 Y = \{ (x, y) \ : \ x \in X, y \in Y \}.
    \end{equation*}
    Show that $X \oplus_1 Y$ is a normed space, with the norm defined as 
    \begin{equation*}
        \|(x,y)\| := \|x\| + \|y\|.
    \end{equation*}
    \item[Proof:] So we mainly need to show the three norm axioms.
    Of course. Here is a clear verification of the norm axioms for the product space $X \oplus_1 Y$.

Let $X$ and $Y$ be normed spaces. We need to verify that $\|(x,y)\| := \|x\|_X + \|y\|_Y$ satisfies the three norm axioms for the space $X \oplus Y$.

Since $\|\cdot\|_X$ and $\|\cdot\|_Y$ are norms on their respective spaces, we know that $\|x\|_X \ge 0$ and $\|y\|_Y \ge 0$. The sum of two non-negative numbers is non-negative, so $\|(x,y)\| = \|x\|_X + \|y\|_Y \ge 0$.

 We must show equivalence with the zero vector $\mathbf{0} = (\mathbf{0}_X, \mathbf{0}_Y)$.
If $(x,y) = \mathbf{0}$, Then $x = \mathbf{0}_X$ and $y = \mathbf{0}_Y$. The norm is $\|(\mathbf{0}_X, \mathbf{0}_Y)\| = \|\mathbf{0}_X\|_X + \|\mathbf{0}_Y\|_Y = 0 + 0 = 0$.
   
    If $\|(x,y)\| = 0$: This means $\|x\|_X + \|y\|_Y = 0$. Since $\|x\|_X \ge 0$ and $\|y\|_Y \ge 0$, their sum can only be zero if both terms are zero individually. Thus, $\|x\|_X = 0$ and $\|y\|_Y = 0$. By the definiteness property on spaces $X$ and $Y$, this implies $x = \mathbf{0}_X$ and $y = \mathbf{0}_Y$. Therefore, $(x,y) = \mathbf{0}$.


Now let's check for homogeneity:
$$\|\alpha(x,y)\| = \|(\alpha x, \alpha y)\| = \|\alpha x\|_X + \|\alpha y\|_Y$$
Using the homogeneity property on $X$ and $Y$:
$$= |\alpha|\|x\|_X + |\alpha|\|y\|_Y = |\alpha| (\|x\|_X + \|y\|_Y) = |\alpha| \|(x,y)\|$$



First, calculate the left side using the definition of vector addition and the product norm:
$$\|(x_1, y_1) + (x_2, y_2)\| = \|(x_1 + x_2, y_1 + y_2)\| = \|x_1 + x_2\|_X + \|y_1 + y_2\|_Y$$
Now, apply the triangle inequality separately to the terms from space $X$ and space $Y$:
$$\|x_1 + x_2\|_X \le \|x_1\|_X + \|x_2\|_X$$
$$\|y_1 + y_2\|_Y \le \|y_1\|_Y + \|y_2\|_Y$$
Substitute these inequalities back into the expression for the sum:
$$\|x_1 + x_2\|_X + \|y_1 + y_2\|_Y \le (\|x_1\|_X + \|x_2\|_X) + (\|y_1\|_Y + \|y_2\|_Y)$$
Finally, rearrange the terms on the right side to match the definition of the product norm for each vector:
$$= (\|x_1\|_X + \|y_1\|_Y) + (\|x_2\|_X + \|y_2\|_Y) = \|(x_1, y_1)\| + \|(x_2, y_2)\|$$
Thus, $\|(x_1, y_1) + (x_2, y_2)\| \le \|(x_1, y_1)\| + \|(x_2, y_2)\|$. \qed


    


    \item A seminorm on a linear vector space $E$ is a function $\| \cdot \| : E \to \mathbb{R}$ which satisfies all norm axioms except the second part of axiom (i). That is, there may exist nonzero vectors $\|x\| = 0$.
   
    Show that one can convert a seminorm into a norm by factoring out the zero directions. Mathematically, show that $kep(p) := {x \in E : \|x\| = 0}$ is a linear subspace of E. Show that the quotient space $E/ker(p)$ is a normed space, with the norm defined as 
    \begin{equation*}
        \|[x]\| := \|x\| ,  x \in E
    \end{equation*}
    Illustrate this procedure by constructing the normed space $L_\infty$ from the semi-normed space of all essentially bounded functions with the essential sup-norm. 

\item[Proof:]So first let's show that $\ker(p)$ is a linear subspace

Let $p(x) = \|x\|$ be the seminorm on space $E$. We need to show that the kernel, $\ker(p) $ satifies of it containing the zero vector, and closed under addition and multiplication
 Using the homogeneity property of the seminorm, we have $\|\mathbf{0}\| = \|0 \cdot x\| = |0| \cdot \|x\| = 0$. Since $\|\mathbf{0}\| = 0$, the zero vector belongs to $\ker(p)$.

Let $x$ and $y$ be two elements in $\ker(p)$. This means $\|x\| = 0$ and $\|y\| = 0$. We check the norm of their sum using the triangle inequality for seminorms:
    $$0 \le \|x + y\| \le \|x\| + \|y\| = 0 + 0 = 0$$
    Since $\|x + y\|$ is bounded above and below by 0, we must have $\|x + y\| = 0$. Therefore, $x + y \in \ker(p)$.

Let $x \in \ker(p)$ (so $\|x\| = 0$) and let $\alpha$ be any scalar. Using the homogeneity property:
    $$\|\alpha x\| = |\alpha| \cdot \|x\| = |\alpha| \cdot 0 = 0$$
    Therefore, $\alpha x \in \ker(p)$.

Since all conditions hold, $\ker(p)$ is a linear subspace of $E$.


Before checking the axioms, we must ensure the norm value $\|[x]\| = \|x\|$ is independent of which representative $x$ we choose from the class $[x]$.

Assume $[x] = [y]$. This implies that $x - y \in \ker(p)$, so $\|x - y\| = 0$. We must show that $\|x\| = \|y\|$.
\begin{equation*}
    

  \|x\| = \|(x - y) + y\| \le \|x - y\| + \|y\| = 0 + \|y\| = \|y\|$. So, $\|x\| \le \|y\|.
\end{equation*}
\begin{equation*}
    
  \|y\| = \|(y - x) + x\| \le \|y - x\| + \|x\| = \|x - y\| + \|x\| = 0 + \|x\| = \|x\|. 
  \end{equation*}
  So, $\|y\| \le \|x\|$.
Since $\|x\| \le \|y\|$ and $\|y\| \le \|x\|$, we conclude $\|x\| = \|y\|$. The norm definition is well-defined.

Now we need to show the norm axioms. Assume $\|[x]\| = 0$. By definition, this means $\|x\| = 0$. By definition of the kernel, this implies $x \in \ker(p)$. The zero element $[\mathbf{0}]$ in the quotient space $E/\ker(p)$ is precisely the set $\ker(p)$. Thus, $[x] = [\mathbf{0}]$.

    $$\|\alpha [x]\| = \|[\alpha x]\| = \|\alpha x\| = |\alpha| \|x\| = |\alpha| \|[x]\|$$
    The properties pass directly from the seminorm on $x$ to the norm on $[x]$.

    $$\|[x] + [y]\| = \|[x+y]\| = \|x+y\| \le \|x\| + \|y\| = \|[x]\| + \|[y]\|$$
    The inequality passes directly from the seminorm on $x$ and $y$ to the norm on $[x]$ and $[y]$.



    \newline
    \textbf{Illustration:}
    Let $E$ be the space of all essentially bounded functions. The function $p(f) = \|f\|_\infty = \operatorname{ess\,sup} |f(x)|$ is a seminorm on $E$.
    The kernel $\ker(p)$ is the set of functions where $\|f\|_\infty = 0$. As we proved in problem 1f, this corresponds exactly to the set of functions where $f(x) = 0$ almost everywhere.
    By forming the quotient space $E/\ker(p)$, we identify all functions that are equal almost everywhere. This process constructs the true normed space $L_\infty$ from the seminormed space of essentially bounded functions.
\qed

    



\end{enumerate}

\end{document}
