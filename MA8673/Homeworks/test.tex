\documentclass[]{article}

% Packages for mathematics
\usepackage{amsmath, amssymb, amsthm}
\usepackage{mathtools}
\usepackage{gensymb}
\usepackage{bm}
\usepackage{authblk}

% Package for graphics
\usepackage{graphicx}
\graphicspath{{../figures/}}
\usepackage{subcaption}
\usepackage{placeins}


% Package for page layout and headers/footers
%\usepackage{geometry}
%\geometry{margin=1in}


% Package for clickable links
\usepackage{hyperref}
\hypersetup{
    colorlinks=true,
    linkcolor=blue,
    citecolor=blue,
    urlcolor=blue,
}

% Package for algorithms
\usepackage{float}
\usepackage{algorithm}
\usepackage{algpseudocode}
\usepackage{matlab-prettifier}
\usepackage{tabularx}
\usepackage{amsmath, physics}

% Custom theorem environments
\newtheorem{theorem}{Theorem}[section]
\newtheorem{lemma}[theorem]{Lemma}
\newtheorem{corollary}[theorem]{Corollary}
\newtheorem{proposition}{Proposition}
\newtheorem{remark}{Remark}
\theoremstyle{definition}
\newtheorem{definition}[theorem]{Definition}

\title{Functional Analysis Homework 13}
\author{Kevin Ho}
\date{\today}

\begin{document}

\maketitle

\begin{enumerate}
    
    
    \begin{proof}Let $\{x_n\}$ be a sequence in $\mathcal{D}(A)$ such that $x_n \to x$ and $Ax_n \to y$ in $\mathcal{H}$. To prove that $A$ is closed, we must show that $x \in \mathcal{D}(A)$ and $Ax = y$.Since $\{Ax_n\}$ is a convergent sequence, it is a Cauchy sequence. By the normality condition $\|Ah\| = \|A^*h\|$, we have for any $n, m$:
$$
\|A^*x_n - A^*x_m\| = \|A^*(x_n - x_m)\| = \|A(x_n - x_m)\| = \|Ax_n - Ax_m\|
$$
Since $\{Ax_n\}$ is Cauchy, the Right Hand Side goes to 0 as $n,m \to \infty$. Therefore, $\{A^*x_n\}$ is also a Cauchy sequence in $\mathcal{H}$ and must converge to some limit $z$.

We now have:
$$
x_n \to x \quad \text{and} \quad A^*x_n \to z
$$
Recall that the adjoint operator $A^*$ is always closed. By the definition of closedness applied to $A^*$, this implies that $x \in \mathcal{D}(A^*)$ and $A^*x = z$.

Since $A$ is normal, we are given that $\mathcal{D}(A) = \mathcal{D}(A^*)$. Therefore, $x \in \mathcal{D}(A)$.

Finally, we must show that $Ax = y$. We apply the norm equality to the vector $x_n - x$:
$$
\|Ax_n - Ax\| = \|A^*(x_n - x)\| = \|A^*x_n - A^*x\|
$$
 taking the limit as $n \to \infty$:
\begin{itemize}
    \item On the right side, since $A^*x_n \to z$ and we found $A^*x = z$, the term $\|A^*x_n - A^*x\| \to 0$.
    \item Therefore, the left side $\|Ax_n - Ax\| \to 0$, which implies $Ax_n \to Ax$.
\end{itemize}

Since we originally assumed $Ax_n \to y$, by the uniqueness of limits, we must have $Ax = y$.

Thus, $x \in \mathcal{D}(A)$ and $Ax = y$, proving that $A$ is closed.
\end{proof}








\begin{proof}
\textbf{Example (domains).}
Let $\mathcal P$ denote the set of complex polynomials. Define
\[
D_1:=\{P(x)e^{-x^2}:P\in\mathcal P\},\qquad
D_2:=\{Q(x)e^{-x^4}:Q\in\mathcal P\}.
\]
Both are linear subspaces of $L^2(\mathbb R)$.

\textbf{Density.}
Consider the weighted Hilbert spaces
\[
H_1 := L^2(\mathbb R, e^{-2x^2}\,dx),\qquad
H_2 := L^2(\mathbb R, e^{-2x^4}\,dx),
\]
and unitary maps $U_1:H_1\to L^2(\mathbb R)$, $U_2:H_2\to L^2(\mathbb R)$ given by
\[
(U_1h)(x)=h(x)e^{-x^2},\qquad (U_2h)(x)=h(x)e^{-x^4}.
\]
Then $D_1 = U_1(\mathcal P)$ and $D_2 = U_2(\mathcal P)$.
Using the standard fact that polynomials are dense in $H_1$ and in $H_2$ (Gaussian-type weights),
it follows that $D_1$ and $D_2$ are dense in $L^2(\mathbb R)$.

\textbf{Disjointness.}
Let $f\in D_1\cap D_2$. Then for some $P,Q\in\mathcal P$,
\[
P(x)e^{-x^2}=Q(x)e^{-x^4}\quad\text{a.e.}
\]
Hence $Q(x)=P(x)e^{x^4-x^2}$ a.e. If $P\not\equiv 0$, then $|P(x)e^{x^4-x^2}|\to\infty$
as $|x|\to\infty$, which is impossible for a polynomial $Q$. Thus $P\equiv 0$ and $Q\equiv 0$,
so $f=0$. Therefore $D_1\cap D_2=\{0\}$.

\textbf{Essential self-adjointness for $M_x$ on $D_1$.}
Let $T_1 := M_x|_{D_1}$, so $T_1f = xf$ with domain $D_1$.
First $D_1\subset \mathcal D(M_x)$ because if $f=P e^{-x^2}$ then $xf=xP e^{-x^2}\in L^2$.
Also $x$ is real-valued, hence $T_1$ is symmetric.

We use the criterion: $T_1$ is essentially self-adjoint iff $\operatorname{Ran}(T_1\pm i)$ is dense in $L^2$.
Fix $h\in L^2(\mathbb R)$ and define
\[
\psi(x):=\frac{h(x)e^{x^2}}{x-i}.
\]
Note that $\psi\in L^2(\mathbb R,(x^2+1)e^{-2x^2}\,dx)$ since
\[
\int_{\mathbb R}|\psi(x)|^2 (x^2+1)e^{-2x^2}\,dx
=\int_{\mathbb R}\frac{|h(x)|^2 e^{2x^2}}{|x-i|^2}(x^2+1)e^{-2x^2}\,dx
=\int_{\mathbb R}|h(x)|^2\,dx <\infty.
\]
By density of polynomials in $L^2(\mathbb R,(x^2+1)e^{-2x^2}\,dx)$, choose polynomials $P_n$ such that
\[
\int_{\mathbb R}\left|P_n(x)-\psi(x)\right|^2 (x^2+1)e^{-2x^2}\,dx \to 0.
\]
Multiplying out shows
\[
\|(x-i)P_n(x)e^{-x^2}-h\|_{L^2}^2
=\int_{\mathbb R}\left|P_n(x)-\psi(x)\right|^2 (x^2+1)e^{-2x^2}\,dx \to 0.
\]
Thus $h$ lies in the closure of $\operatorname{Ran}(T_1-i)$, so $\operatorname{Ran}(T_1-i)$ is dense in $L^2$.
The same argument works with $x+i$, hence $\operatorname{Ran}(T_1+i)$ is also dense. Therefore $T_1$ is essentially self-adjoint.

\textbf{Essential self-adjointness for $M_{x^2}$ on $D_2$.}
Let $T_2 := M_{x^2}|_{D_2}$, so $T_2f = x^2f$ with domain $D_2$.
Again $D_2\subset \mathcal D(M_{x^2})$ and $T_2$ is symmetric since $x^2$ is real.
Repeat the previous density-of-range argument with
\[
\phi(x):=\frac{h(x)e^{x^4}}{x^2-i},
\]
and the weight $(x^4+1)e^{-2x^4}$ (using $|x^2-i|^2=x^4+1$). This gives density of
$\operatorname{Ran}(T_2\pm i)$, hence $T_2$ is essentially self-adjoint.

Therefore $M_x$ is essentially self-adjoint on $D_1$ and $M_{x^2}$ is essentially self-adjoint on $D_2$,
with $D_1\cap D_2=\{0\}$.
\end{proof}

\end{enumerate}

\end{document}