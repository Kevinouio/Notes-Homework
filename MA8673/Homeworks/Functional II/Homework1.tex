\documentclass[]{article}

% Packages for mathematics
\usepackage{amsmath, amssymb, amsthm}
\usepackage{mathtools}
\usepackage{gensymb}
\usepackage{bm}
\usepackage{authblk}

% Package for graphics
\usepackage{graphicx}
\graphicspath{{../figures/}}
\usepackage{subcaption}
\usepackage{placeins}


% Package for page layout and headers/footers
%\usepackage{geometry}
%\geometry{margin=1in}

\usepackage{mathrsfs}
% Package for clickable links
\usepackage{hyperref}
\hypersetup{
    colorlinks=true,
    linkcolor=blue,
    citecolor=blue,
    urlcolor=blue,
}

% Package for algorithms
\usepackage{float}
\usepackage{algorithm}
\usepackage{algpseudocode}
\usepackage{matlab-prettifier}
\usepackage{tabularx}
\usepackage{amsmath, physics}

% Custom theorem environments
\newtheorem{theorem}{Theorem}[section]
\newtheorem{lemma}[theorem]{Lemma}
\newtheorem{corollary}[theorem]{Corollary}
\newtheorem{proposition}{Proposition}
\newtheorem{remark}{Remark}
\theoremstyle{definition}
\newtheorem{definition}[theorem]{Definition}

\title{Functional Analysis (MA 8673) Homework 1}
\author{Kevin Ho}
\date{\today}

\begin{document}

\maketitle

\begin{enumerate}
    \item Define 

    $$
    A = i\frac{d}{dx}
    $$
    on $L^2$[0,1]. Define two domains for $A: \mathcal{D}_1 (A) = AC[0,1]$ and $\mathcal{D}_2(A) = \{f \in AC[0,1] : f(0) = 0\}.$ Prove that $A$ is closed on $\mathcal{D}_1(A)$. Prove or disprove A is closed on $\mathcal{D}_2(A)$

    \begin{proof}
    Let $\{f_n\} \in AC[0,1]$ such that ${f_n}\rightarrow f \in  L^2[0,1]$ and $Af_n \rightarrow g \in L^2$. Note that $Af_n = if_n'$ and so $if_n' \rightarrow g = f_n'\rightarrow -ig$.
    
    For simplicity let $-ig = h.$ Another thing to note, since we are in a finite space of $[0,1]$, by Holder's inequality, $f'_n\rightarrow h\in L^1[0,1]$. So now let's observe the integration of the sequence of $f_n$. By FTOC we get,

    \begin{equation}
    \label{1}
         \int_0^x f'_n(t)dt = f_n(x)-f_n(0) 
    \end{equation}
   


    $$
    \rightarrow f_n(0) = f_n(x) -  \int_0^x f'_n(t)dt
    $$
    Since we know this converges in $L^1$, we integrate both sides w.r.t. x to get 

    $$
    \int_0^1 f_n(0)dx = \int_0^1f_n(x)dx -  \int_0^1\int_0^x f'_n(t)dtdx
    $$
    
    $$
    \rightarrow f_n(0) = \int_0^1f_n(x)dx -  \int_0^1\int_0^x f'_n(t)dtdx
    $$
   
    Let $c = \lim_{n \to \infty} f_n(0)$. (We know this limit exists because the RHS terms converge).
    
    Now let's go back to \eqref{1} to get the relation
    $$ f_n(x) = f_n(0) + \int_0^x f_n'(t) dt $$
    Taking the limit as $n \to \infty$:
    $$ f(x) = c + \int_0^x h(t) dt $$
    
    This equation tells us two things:
    \begin{enumerate}
        \item The function $f$ is absolutely continuous (since it is an integral of an $L^1$ function), so $f \in \mathcal{D}_1(A)$.
        \item Differentiating both sides gives $f'(x) = h(x)$.
    \end{enumerate}
    
    Recall that $h = -ig$. Therefore:
    $$ f' = -ig \implies i f' = g \implies Af = g $$
    
    Thus, $A$ is closed on $\mathcal{D}_1(A)$.
    
    So now let's observe  $\{f_n\} \in D_2(A)$. Assume all the earlier assumptions as shown for $D_1(A)$. Since $f_n \in D_2(A),$ we have $f_n(0) = 0$. By FTOC we have
    $$
    f_n(x) = \int^x_0f'_n(t)dt.
    $$

    Now for any $x \in [0,1]$, we bound the difference by 
    $$
    |f_n(x) - \int_0^xh(t)dt| \leq \int^x_0|f'_n(t) -h(t)|dt \leq \|f'_n-h\|_{L^1}
    $$

    Now assume $f_n\in \mathcal D_2(A)$, so $f_n(0)=0$ for all $n$.
Hence $c=\lim_{n\to\infty} f_n(0)=0$, and so the function
\[
F(x)=c+\int_0^x h(t)\,dt
\]
satisfies $F(0)=0$, i.e. $F\in\mathcal D_2(A)$.
From the estimate above we already have $f_n\to F$ in $L^2$, hence $f=F$ a.e.
Also $F'=h=-ig$ a.e., so $Af=g$. Therefore $A$ is closed on $\mathcal D_2(A)$.

    
   
    \end{proof}

    \item Let $A$ and $B$ be operators defined on a Hilbert space $\mathcal{H}$. 
    Show that $(\alpha A)^* = \bar{\alpha}A^*$ for scalar $\alpha$.
    Moreover show $A^* + B^* \subseteq (A+B)^*$ where $\mathcal{D}(A^* + B^*) = \mathcal{D}(A^*) \cap \mathcal{D}(B^*).$ 
    Show that $(A+B)^* = A^* + B^*$ only if one of the operators is bounded. Give an example where equality doesn't hold.

    \begin{proof}
        Define $x,y \in \mathcal{H}$ s.t. $x\in \mathcal D(A)$ and $y\in \mathcal D(A^*) $ and $\alpha$ be a scalar. We can manipulate
        $
        \langle (\alpha A) x,y  \rangle 
        $ as follows.
         
        $$
        \alpha \langle  x,A^*y  \rangle =  \langle  x,\bar{\alpha}A^*y  \rangle 
        $$
        This implies that $y \in ((\alpha A)^*)$ and so $(\alpha A)^*y = \bar \alpha A^*y$

        Now let's show that  $A^* + B^* \subseteq (A+B)^*$.

         \begin{align*}
         \langle  (A+B)x,y  \rangle &= \langle  Ax,y  \rangle +  \langle  Bx,y  \rangle \\ 
         &= \langle  x,A^*y  \rangle + \langle  x,B^*y  \rangle \\
         &= \langle  x,(A+B)^*y  \rangle 
        \end{align*}

        This holds $\forall x\in \mathcal D(A+B),$ it follows that $y \in ((A+B)^*)$ and $(A+B)^*y = (A^*+B^*)y .$


        Assume $A$ is bounded. Then $\mathcal{D}(A) = \mathcal{H}$, which implies $\mathcal{D}(A^*) = \mathcal{H}$.
Consequently, the domain of the sum simplifies to $\mathcal{D}(A+B) = \mathcal{D}(B)$.

From Part 2, we already know $A^* + B^* \subseteq (A+B)^*$. We must now show the reverse inclusion $(A+B)^* \subseteq A^* + B^*$.

Let $y \in \mathcal{D}((A+B)^*)$. By definition, there exists a vector $z \in \mathcal{H}$ such that for all $x \in \mathcal{D}(A+B) = \mathcal{D}(B)$:
$$
\langle (A+B)x, y \rangle = \langle x, z \rangle
$$
Using the linearity of the inner product, we expand the left side:
$$
\langle Ax, y \rangle + \langle Bx, y \rangle = \langle x, z \rangle
$$
Since $A$ is bounded and defined on all of $\mathcal{H}$, we know that $y \in \mathcal{D}(A^*)$ so $\langle Ax, y \rangle = \langle x, A^*y \rangle$. Substituting this into the equation, we get
$$
\langle x, A^*y \rangle + \langle Bx, y \rangle = \langle x, z \rangle
$$
$$
\langle Bx, y \rangle = \langle x, z \rangle - \langle x, A^*y \rangle
$$
$$
\langle Bx, y \rangle = \langle x, z - A^*y \rangle
$$
This equation holds for all $x \in \mathcal{D}(B)$. By the definition of the adjoint, this implies that $y \in \mathcal{D}(B^*)$ and $B^*y = z - A^*y$.

Since $y \in \mathcal{D}(B^*)$ and $y \in \mathcal{D}(A^*)$, we have $y \in \mathcal{D}(A^*) \cap \mathcal{D}(B^*) = \mathcal{D}(A^* + B^*)$.

Thus, $(A+B)^* \subseteq A^* + B^*$ and so we have $(A+B)^* = A^* + B^*$.


        \textbf{Counter Example: }An example of where the inequality doesn't hold is let $A=-B.$ Also let $A$ be an unbounded operator. 
        We have the following to occur of $A+B = 0$ on $D(A),$ so $(A+B)^* = 0$ on the entire space $\mathcal{H}$. Now observing the adjoint of $(A+B)^*$, we get $A^*+B^* = A^* - A^* = 0$. This implies that the domain is restricted to $\mathcal D(A^*)$. This implies that the domain of $A^*+B^*$ is restricted to $\mathcal{D}(A^*)$. However, the domain of $(A+B)^*$ is the entire space $\mathcal{H}$. This mismatch between $\mathcal{D}(A^*+B^*)$ and $\mathcal{D}((A+B)^*)$ shows that equality does not hold.
        

        

    
    \end{proof}
    

    \item Let $A$ and $B$ be operators defined on a Hilbert space $\mathcal{H}$ such that $AB$ is densely defined. Prove that $(AB)^* \supset B^*A^*.$ Moreover if $B$ is bounded then show $(BA)^*=A^*B^*.$
    \begin{proof}
        Let $y \in \mathcal D(A^*)$ and  $A^*y \in \mathcal D(B^*)$. Let $x\in \mathcal D(AB)$ so that we have $Bx\in \mathcal D(A)$. We then have the following 
        $$
         \langle  (AB)x,y  \rangle  = \langle  A(Bx),y  \rangle  = \langle  Bx,A^*y  \rangle
        $$
        $$
        \langle  x,B^*A^*y  \rangle
        $$


        This shows that $y\in \mathcal D((AB)^*)$ and that $(AB)^* \supset B^*A^*$. Now we then want to show the other direction of $(AB)^* \subset B^*A^*.$
        
        Let $y\in (BA)^*$, then $\exists z\in \mathcal H $ such that 
        $$
        \langle BA x, y\rangle = \langle  x, z\rangle 
        $$
        We can rewrite as follows:
        $$
        \langle  Ax,B^* y\rangle = \langle x,z \rangle, \forall x \in \mathcal{D}(A). 
        $$
        This means that $B^*y \in \mathcal{D}(A^*) $ and $ A^*(B^*y) = z = (BA)^*y.$ Leading to the equality of $(BA)^*=A^*B^*.$
        
    \end{proof}

    \item An alternative way to define a normal operator (to allow for unboundedness) is the following: $A$ is called \textbf{normal} if $\|Af\| = \|A^*f\|$ for all $f \in \mathcal{D}(A) = \mathcal{D}(A^*)$. Prove that if $A$ is normal then so is $A+z$ for all $z \in \mathbb{C}$
    \begin{proof}

    So to satisfy this, we must show that $ \mathcal{D}(A+z) = \mathcal{D}((A+z)^*)$ and $\|(A+z)f\| = \|(A+z)^*f\|, \forall f\in \mathcal D$. 
    
    So first let's show the domain equality.  
    $$ 
    (A+z)^* = A^* + \bar{z}
    $$
    Thus, the domain is given as follows.

    $$
    \mathcal{D}((A+z)^*) = \mathcal{D}(A^* + \bar z) = \mathcal D(A^*)
    $$
    Since $A$ is normal, $\mathcal D(A+z) =\mathcal D(A) =\mathcal D(A^*) =  \mathcal D(
    (A+z)^*)$

    Now we want to show the Norm equality. Let $f \in \mathcal D(A)$. Let's expand out both sides of the equality
    $$
    \|(A+z)f\| = \|(A+z)^*f\|
    $$

    LHS:
    \begin{align*}
        |(A+z)f\|^2  &= \langle (A+z)f, (A+z)f \rangle \\
        &=\langle Af+zf, Af+zf \rangle  \\
        &= \langle Af, Af\rangle + \langle Af, zf\rangle + \langle zf, Af\rangle +\langle zf, zf\rangle \\
        &=\|Af\|^2 + \bar{z}\langle Af,f\rangle + {z}\langle Af,f\rangle + |z|^2\|f\|^2
    \end{align*}


     RHS:
    \begin{align*}
        |(A+z)^*f\|^2  &= \langle (A+z)^*f, (A+z)^*f \rangle \\
        &=\langle A^*f+\bar{z}f, A^*f+\bar{z}f \rangle  \\
        &= \langle A^*f, A^*f\rangle + \langle A^*f, \bar{z}f\rangle + \langle \bar{z}f, A^*f\rangle +\langle \bar{z}f, \bar{z}f\rangle \\
        &=\|A^*f\|^2 + {z}\langle A^*f,f\rangle + \bar{z}\langle A^*f,f\rangle + |z|^2\|f\|^2
    \end{align*}
    Note that both sides expand to the same term except the first term of both. Note that $A$ is normal. With that $\|Af\|^2 = \|A^*f\|^2$. Thus $A+z$ is normal.

        
    \end{proof}
        % complex ocnjucation or polarization identity most likely so just like computational stuff
        

    \item Using the above definition, prove that normal operators are always closed.
    \begin{proof}Let $\{x_n\}$ be a sequence in $\mathcal{D}(A)$ such that $x_n \to x$ and $Ax_n \to y$ in $\mathcal{H}$. To prove that $A$ is closed, we must show that $x \in \mathcal{D}(A)$ and $Ax = y$. Since $\{Ax_n\}$ is a convergent sequence, it is a Cauchy sequence. By the normality condition $\|Ah\| = \|A^*h\|$, we have for any $n, m$:
$$
\|A^*x_n - A^*x_m\| = \|A^*(x_n - x_m)\| = \|A(x_n - x_m)\| = \|Ax_n - Ax_m\|
$$
Since $\{Ax_n\}$ is Cauchy, $ \|Ax_n - Ax_m\|\rightarrow0$ as $n,m \to \infty$. Therefore, $\{A^*x_n\}$ is also a Cauchy sequence in $\mathcal{H}$ and must converge to some limit $z$.

We now have:
$$
x_n \to x \quad \text{and} \quad A^*x_n \to z
$$
Recall that the adjoint operator $A^*$ is always closed. By the definition of closedness applied to $A^*$, this implies that $x \in \mathcal{D}(A^*)$ and $A^*x = z$.

Since $A$ is normal, we are given that $\mathcal{D}(A) = \mathcal{D}(A^*)$. Therefore, $x \in \mathcal{D}(A)$.

Finally, we must show that $Ax = y$. We apply the norm equality to the vector $x_n - x$ to get
$$
\|Ax_n - Ax\| = \|A^*(x_n - x)\| = \|A^*x_n - A^*x\|.
$$
When we take the limit as $n \to \infty$, on the right side, since $A^*x_n \to z$ and we found $A^*x = z$, the term $\|A^*x_n - A^*x\| \to 0$. Thus the left side $\|Ax_n - Ax\| \to 0$, which implies $Ax_n \to Ax$.

Since we originally assumed $Ax_n \to y$, by the uniqueness of limits, we must have $Ax = y$.

Thus, $x \in \mathcal{D}(A)$ and $Ax = y$, proving that $A$ is closed.
\end{proof}
    
    


    

\end{enumerate}

\end{document}
