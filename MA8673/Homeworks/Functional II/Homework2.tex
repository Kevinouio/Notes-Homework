\documentclass[]{article}

% Packages for mathematics
\usepackage{amsmath, amssymb, amsthm}
\usepackage{mathtools}
\usepackage{gensymb}
\usepackage{bm}
\usepackage{authblk}

% Package for graphics
\usepackage{graphicx}
\graphicspath{{../figures/}}
\usepackage{subcaption}
\usepackage{placeins}


% Package for page layout and headers/footers
%\usepackage{geometry}
%\geometry{margin=1in}

\usepackage{mathrsfs}
% Package for clickable links
\usepackage{hyperref}
\hypersetup{
    colorlinks=true,
    linkcolor=blue,
    citecolor=blue,
    urlcolor=blue,
}

% Package for algorithms
\usepackage{float}
\usepackage{algorithm}
\usepackage{algpseudocode}
\usepackage{matlab-prettifier}
\usepackage{tabularx}
\usepackage{amsmath, physics}

% Custom theorem environments
\newtheorem{theorem}{Theorem}[section]
\newtheorem{lemma}[theorem]{Lemma}
\newtheorem{corollary}[theorem]{Corollary}
\newtheorem{proposition}{Proposition}
\newtheorem{remark}{Remark}
\theoremstyle{definition}
\newtheorem{definition}[theorem]{Definition}

\title{Functional Analysis (MA 8673) Homework 1}
\author{Kevin Ho}
\date{\today}

\begin{document}

\maketitle

\begin{enumerate}
    \item Define 
    $$
    A = -i\frac{d}{dx}
    $$
    on $L^2[0,2\pi]$ with domain $\mathcal{D}(A) = \{f\in C^1[0,2\pi]: f(0) = f(2\pi) =0\}.$ Prove that $A$ is symmetric. Find $A^*$ and its respective domain $\mathcal{D}(A^*)$. Is $A$ self-adjoint? If not find a self-adjoint extension of $A$.


    \begin{proof}  
    \textbf{Symmetry}: let $f,g \in \mathcal{D}(A) $. Then we have the following derivation of the inner product between $f$ and $g$ be,

    \begin{align*}
         \langle    Af,g   \rangle &=  \int_0^{2\pi}-if'(x)\overline{g(x)}dx, \text{ Integration by Parts} \\ 
         &=-i\left(f(x)\overline{g(x)}\rvert_0^{{2\pi}} -\int_0^{2\pi}f(x)\overline{g'(x)}dx\right)\\
         &=-i\left(\left(f({2\pi})\overline{g({2\pi})})- f(0)\overline{g(0)}\right) -\int_0^{2\pi}f(x)\overline{g'(x)}dx\right)\\
         &= \int_0^{2\pi}f(x)\overline{ig'(x)})dx \\
         &= \int_0^{2\pi}f(x)(-i\overline{g'(x)}))dx \\
         &= \langle    f,Ag   \rangle 
    \end{align*}
    Thus A is symmetric.
    

    

    \textbf{    $A^* \text{ and } \mathcal D({A^*}) $}: So note in the previous part the term 
    $$
    -i\left(f({2\pi})\overline{g({2\pi})})- f(0)\overline{g(0)}\right)
    $$
    Now note that there is no necessary conditions that is restricted on $g$'s boundary conditions. We then end with the following of
    $$
    -i\left((0)\overline{g({2\pi})})- (0)\overline{g(0)}\right)
    $$
    This gives us that $g$ isn't constrained to the conditions that $f$ is given in the Domain $\mathcal{D}(A)$ where $f(0) = f(2\pi)=0$. 
    With this in mind, this gives us properties to which $g$ must satisfy being that it must be absolutely continuous on the interval to ensure that FTOC holds and that $g' \in L^2[0,2\pi]$, giving us that the domain must be square-integrable. This proves that $A$ is not a self adjoint operator. Now defining $A^*$, we simply look at our reconstruction of $\langle Af,g \rangle$ to get that $A^* = \langle f,-ig' \rangle$. 


    \textbf{Self-Adjoint Extension}: Since $A$ is not self-adjoint we try to find the extension to have A to be self-adjoint. A simple way to do this is since the domain do not match, we can just extend $\mathcal{D}(A)$ to be $\{f\in AC[0,2\pi]:f(0)=f(2\pi)\}$ and it is clear that the domains now match as above. 

        
    \end{proof}



    \item (a) Suppose that $B$ is a symmetric operator such that $A\subset B$ and that $\text{Ran}(A+i) = \text{Ran}(B+i)$. Prove that $B=A$.
    
    (b) Suppose that $A$ is a symmetric operator with $\text{Ran}(A+i) = \mathcal{H}$
         but $\text{Ran}(A-i) \neq \mathcal{H}$. Prove that $A$ has no self-adjoint extension.
    
    \begin{proof}
        \textbf{(a):} So we know that $\mathcal{D}(A) \subset \mathcal{D}(B)$ from the given statement so we want to show the other direction. 
        
        Let $g \in \mathcal{D}(B)$. Since $\text{Ran}(B+i) = \text{Ran}(A+i)$, there exists $f \in \mathcal{D}(A)$ such that:
$$(B+i)g = (A+i)f$$
Since $A \subset B$, $$(B+i)g = (B+i)f.$$ Thus:
$$(B+i)(g-f) = 0$$
Since $B$ is symmetric, its eigenvalues are real, so $\text{Ker}(B+i)=\{0\}$. Therefore $g-f=0$, implying $g=f \in \mathcal{D}(A)$. 
Since $\mathcal{D}(B) \subseteq \mathcal{D}(A)$, we have $B=A$.
        
        \textbf{(b):}
        Assume for contradiction that a self-adjoint extension $\tilde{A}$ exists ($A \subset \tilde{A}$).
Since $\tilde{A}$ is self-adjoint, $\text{Ran}(\tilde{A} \pm i) = \mathcal{H}$.
We are given $\text{Ran}(A+i) = \mathcal{H}$. Thus:
$$\text{Ran}(A+i) = \text{Ran}(\tilde{A}+i) = \mathcal{H}$$
By the result in part (a), this implies $A = \tilde{A}$.
However, this would mean $A$ is self-adjoint, which requires $\text{Ran}(A-i) = \mathcal{H}$. This contradicts the given condition that $\text{Ran}(A-i) \neq \mathcal{H}$.
Thus, no such extension exists.
        
    \end{proof}

    \item Define 
    $$
    A = -\frac{d^2}{dx^2}
    $$
    on $L^2[0,2\pi]$ with domain $\mathcal{D}(A) =C^\infty_0(\mathbb{R}).$ Compute $A^*$ and its respective domain $\mathcal{D}(A^*)$. Is $A$ essentially self-adjoint?

    \begin{proof}
        \textbf{$A^* \text{ and } D(A^*)$:}
        Let $f\in \mathcal{D}(A)$ and $g\in \mathcal{D}(A^*)$, We decompose the Operator as follows.




        \begin{align*}
            \langle Af, {g} \rangle &= \int_0^{2\pi} -f''(x)\overline{g(x)}dx ,\text{ Integration by Parts}\\
            &= -\left( f'(x)\overline{g(x)}|_0^{2\pi}-  \int_0^{2\pi} f'(x)\overline{g'(x)}dx\right), \text{ Integration by Parts} \\
            &= -\left( f'(x)\overline{g(x)}|_0^{2\pi}- f(x)\overline{g'(x)}|_0^{2\pi} + \int_0^{2\pi} f(x)\overline{g''(x)}dx\right), \text{ Compact Support of $f$} \\
            &= -\int_0^{2\pi} f(x)\overline{g''(x)}dx\\
            &= \langle f, -g'' \rangle
        \end{align*}
        Thus showing that $A^* = -\frac{d^2}{dx^2}$. Now for $\mathcal D(A^*)$, similar to the argument as stated in the first question, $g$ is not in the same domain as $f$ as it does not require the compact support that $f$ has, and so $D(A^*) =\{ g \in L^2[0,2\pi]: g \text{ and } g' \text{ are absolutely continuous and  }  g'' \in L^2[0,2\pi]\}$ 

        

          \textbf{Ess Self Adjoint:} My claim is that the operator $A$ is not essentially Self Adjoint. To prove this claim, let's show that the closure of $A$ does not equal the adjoint. So note that for functions in $\mathcal D(A)$, it requires them to have compact support leading to a zero boundary condition. Taking the closure of the operator extends the domain to functions with valid derivatives, and conserving the boundary conditions being zero. 
          
          However, as we have shown for the domain of $\mathcal D(A^*)$, it has no strict conditions on the boundaries of the functions in the domain. This fails the property of $\mathcal D(\bar A) = \mathcal D(A^*)$ so thus the closure is not self-adjoint. Therefore not essentially self-adjoint.
    \end{proof}



   item Let $M_g$ be the multiplication operator by $g$. Find two dense linear subspaces of $L^2(\mathbb{R}), \mathcal D_1 \text{ and }\mathcal D_2 $ with $\mathcal D_1 \cap \mathcal D_2 = \{0\}$ so that $M_x$ is essentially self-adjoint on $\mathcal  D_1$ and $M_{x^2}$ is essentially self-adjoint on $\mathcal D_2$.

    \begin{proof}
    \textbf{Example:} So let's define both of our Domains 
    
    \begin{align*}
        &\mathcal{D}_1 = \{ P(x)e^{-x^2}: P \in \mathcal{P}  \}, \text{$\mathcal{P}$ is the set of all polynomials} \\
        &\mathcal{D}_2 = \{ Q(x)e^{-x^4}: Q \in \mathcal P\}
    \end{align*}
    \textbf{Dense:} The set $\mathcal{D}_1$ contains the linear span of the Hermite functions $\{H_n(x)e^{-x^2/2}\}_{n=0}^\infty$. Since the Hermite functions form an orthonormal basis for $L^2(\mathbb{R})$ (Shown last semester in homework), their finite linear span is dense. For $\mathcal{D}_2$: Since $e^{-x^4}$ is a strictly positive, rapidly decaying weight, the set of weighted polynomials is similarly dense in $L^2(\mathbb{R})$ 
    
    \textbf{Disjoint:} Let $f\in \mathcal D_1 \cap \mathcal D_2$. Then 
    \begin{align*}
    f(x) = P(x)e^{-x^2}&=Q(x)e^{-x^4} \\
    \rightarrow P(x)e^{x^4-x^2}&=Q(x)
    \end{align*}
    If $P(x) \neq 0$, then $Q(x)$ explodes to infinity which should not happen, thus the polynomials must be the trivial solution $\{0\}$. Therefore $\mathcal D_1 \cap \mathcal D_2 = \{0\}$  
    
    
    \textbf{Essentially Self Adjoint:} A symmetric operator $T$ is essentially self-adjoint on a domain $\mathcal{D}$ if and only if $\text{Ran}(T \pm i)$ is dense in the Hilbert space. Equivalent, we show that $\text{Ran}(T \pm i)^\perp = \{0\}$.
    
    $M_x$ on $\mathcal{D}_1$Let $h \in L^2(\mathbb{R})$ be orthogonal to $\text{Ran}(M_x - i)$. Then for all $f \in \mathcal{D}_1$:$$\langle h, (M_x - i)f \rangle = 0$$$$\rightarrow \int_{-\infty}^{\infty} \overline{h(x)} (x - i) P(x)e^{-x^2} \, dx = 0 \quad \forall P \in \mathcal{P}$$Let $k(x) = \overline{h(x)}(x-i)$. The condition implies that $k$ is orthogonal to the dense set $\mathcal{D}_1$. Therefore, $k(x) = 0$ almost everywhere.$$\overline{h(x)}(x - i) = 0 \quad \text{a.e.}$$Since $(x - i) \neq 0$ for any $x \in \mathbb{R}$, we must have $h(x) = 0$ a.e.The same argument holds for $M_x + i$. Thus, $M_x$ is essentially self-adjoint on $\mathcal{D}_1$. The same argument applies for $\mathcal D_2$ thus $M_{x^2}$ is essentially self-adjoint on $\mathcal D_2$
    
    \end{proof}

\end{enumerate}

\end{document}
