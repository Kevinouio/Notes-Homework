\documentclass[]{article}

% Packages for mathematics
\usepackage{amsmath, amssymb, amsthm}
\usepackage{mathtools}
\usepackage{gensymb}
\usepackage{bm}
\usepackage{authblk}

% Package for graphics
\usepackage{graphicx}
\graphicspath{{../figures/}}
\usepackage{subcaption}
\usepackage{placeins}


% Package for page layout and headers/footers
%\usepackage{geometry}
%\geometry{margin=1in}

\usepackage{mathrsfs}
% Package for clickable links
\usepackage{hyperref}
\hypersetup{
    colorlinks=true,
    linkcolor=blue,
    citecolor=blue,
    urlcolor=blue,
}

% Package for algorithms
\usepackage{float}
\usepackage{algorithm}
\usepackage{algpseudocode}
\usepackage{matlab-prettifier}
\usepackage{tabularx}
\usepackage{amsmath, physics}

% Custom theorem environments
\newtheorem{theorem}{Theorem}[section]
\newtheorem{lemma}[theorem]{Lemma}
\newtheorem{corollary}[theorem]{Corollary}
\newtheorem{proposition}{Proposition}
\newtheorem{remark}{Remark}
\theoremstyle{definition}
\newtheorem{definition}[theorem]{Definition}

\title{Functional Analysis (MA 8673) Homework 1}
\author{Kevin Ho}
\date{\today}

\begin{document}

\maketitle

\begin{enumerate}
    \item Suppose $A$ is self-adjoint and $B$ is any operator such that $\|B-z_0\| \leq r$ for some $z_0 \in \mathbb{C}$ and $r > 0.$ Show that $\sigma(A+B) \subseteq \sigma(A)+ \overline{B_r(z_0)}$ where $ \overline{B_r(z_0)}$ is the ball centered at $z_0$ with radius $r$.

    \begin{proof}
        Let being by observing the properties of the operator $A+B-\lambda I$ with $\lambda \in \sigma(A+B)$. Note that this operator is not invertible due to the $\lambda$ term. So let's rearrange the operator. 

    \begin{align*}
    A+B-\lambda I &= A + B - \lambda I +z_0I-z_0I \\
    &=  (A-\lambda I +z_0I) + (B-z_0I) \\
    &=  (A - (\lambda - z_0  )I)  + (B-z_0I) 
    \end{align*}
    With this rearrangement, let's now observe the spectral radius of both of the operators. First let's look at the $A$ operator and for simplicity sake, let $\lambda  - z_0 = \mu$. So we need to look at two cases of whether $\mu $ is in the spectrum of $A$ or not.

    \textbf{Case 1: ($\mu \in \sigma(A)$)} So this case is trivial since if $\mu$ is in the spectrum of A, then the distance to the spectrum is 0. 


    
    
    \textbf{Case 2: ($\mu \notin \sigma(A)$)} If $\mu$ is not in the spectrum of A, then then we know the operator $A - \mu I$ is an invertible operator. With that in mind, we go back to the original operator to factor as shown below.

\begin{align*}
        A+B-\lambda I &= (A - \mu I)[I  + (A - \mu I)^{-1}(B-z_0I)]
\end{align*}
    From this we find that the operator $[I  + (A - \mu I)^{-1}(B-z_0I)]$ is not invertible. Since it is not invertible, then we also know that the norm (Von Neumann's Series Theorem) is greater than 1. With taht we get the following.

    $$
    \|(A - \mu I)^{-1}(B-z_0I)\| \geq 1
    $$
    With our stated fact about the operator $B$ being  $\|B-z_0\| \leq r$ for some $z_0 \in \mathbb{C}$, we have the following inequality.


    \begin{align*}
         \|(A - \mu I)^{-1}(B-z_0I)\| \geq 1 &\rightarrow  \|(A - \mu I)^{-1}\|r \geq 1 \\ 
        &\rightarrow \|(A - \mu I)^{-1}\| \geq \frac{1}{r}
    \end{align*}


    So now I want to the rewrite the norm on the LHS. 
    
    \textbf{Claim:} $\|(A - \mu I)^{-1}\| = \frac{1}{dist(\mu,\sigma(A))}$.

    So first, note that $A$ is a self-adjoint operator. With that in mind, we can rewrite the norm as below. 

    
   $$ 
    \|(A - \mu I)^{-1}\|  = \sup_{\lambda\in\sigma(A)}|\frac{1}{\lambda - \mu}|
   $$
   by spectral mapping theorem, we can do the following.
   
    $$ 
    \sup_{\lambda\in\sigma(A)}|\frac{1}{\lambda - \mu}| = {\lambda\in\sigma(A)}\frac{1}{\inf_{\lambda\in\sigma(A)}|\lambda - \mu|} = \frac{1}{dist(\mu,\sigma(A))}
   $$

   With this we have shown our claim and now we plug that into our earlier inequality to get

   
   \begin{align*}
         \|(A - \mu I)^{-1}\| \geq \frac{1}{r} &\rightarrow \frac{1}{dist(\mu,\sigma(A))} \geq \frac{1}{r} \\
            &\rightarrow {dist(\mu,\sigma(A))}   \leq r   
    \end{align*}
   
    This implies that there exists an $a\in \sigma(A)$ s.t. $|\mu - a| \leq r$. Let $w = z_0 + (\mu -a)$. Then $|w - z_0 | = |\mu - a| \leq r$ and so $w \in \overline{B_r(z_0)}.$ And 
    $$
    \lambda = z_0+\mu = a + (z_0(\mu-a)) 
    = a + w \in \sigma(A) + \overline{B_r(z_0)}
    $$
    
    

        
    \end{proof}



    \item Let $A$ be an self-adjoint operator and let $P_A$ be its corresponding projection valued measures. Prove that:
    $$
    \sigma(A) = \{  \lambda \in \mathbb{R}: P_A(\lambda-\epsilon, \lambda + \epsilon) \neq0, \forall\epsilon>0 \}.
    $$

    \begin{proof}


\noindent\textbf{($\Rightarrow$)} Assume there exists $\varepsilon>0$ such that
\[
P_A\bigl((\lambda-\varepsilon,\lambda+\varepsilon)\bigr)=0.
\]
Fix $\psi\in\mathcal D(A)$. By the spectral theorem,
\[
\|(A-\lambda I)\psi\|^2=\int_{\mathbb R}|t-\lambda|^2\,d\mu_\psi(t),
\qquad
\mu_\psi(E):=\langle \psi, P_A(E)\psi\rangle.
\]
Since $P_A((\lambda-\varepsilon,\lambda+\varepsilon))=0$, we have
$\mu_\psi((\lambda-\varepsilon,\lambda+\varepsilon))=0$, hence the integral is supported on
$\{t:\ |t-\lambda|\ge \varepsilon\}$. Therefore
\begin{align*}
\|(A-\lambda I)\psi\|^2
&=\int_{|t-\lambda|\ge \varepsilon}|t-\lambda|^2\,d\mu_\psi(t) \\
&\ge \int_{|t-\lambda|\ge \varepsilon}\varepsilon^2\,d\mu_\psi(t) \\
&=\varepsilon^2\int_{\mathbb R}1\,d\mu_\psi(t)\\
&=\varepsilon^2\|\psi\|^2,
\end{align*}
so
\begin{equation}\label{eq:lowerbound}
\|(A-\lambda I)\psi\|\ge \varepsilon\|\psi\|\qquad(\forall\,\psi\in\mathcal D(A)).
\end{equation}

To conclude $\lambda\in\rho(A)$, we construct the inverse using functional calculus.
Define a bounded Borel function
\[
g(t)=
\begin{cases}
\dfrac{1}{t-\lambda}, & |t-\lambda|\ge \varepsilon,\\[6pt]
0, & |t-\lambda|<\varepsilon.
\end{cases}
\]
Then $\|g\|_\infty\le 1/\varepsilon$, hence $B:=g(A)$ is a bounded operator and
$\|B\|\le 1/\varepsilon$. Let $f(t)=t-\lambda$, so $f(A)=A-\lambda I$.
Since $P_A((\lambda-\varepsilon,\lambda+\varepsilon))=0$, we have $g(t)=1/(t-\lambda)$
$P_A$-a.e., and thus $f(t)g(t)=1$ $P_A$-a.e. By the multiplicative property of the
functional calculus,
\[
(A-\lambda I)B=f(A)g(A)=(fg)(A)=I.
\]
Similarly,
\[
B(A-\lambda I)=I
\]
(on $\mathcal D(A)$, the natural domain of $A-\lambda I$). Hence $B=(A-\lambda I)^{-1}$
is a bounded everywhere-defined inverse, so $\lambda\in\rho(A)$.

\medskip
\noindent\textbf{($\Leftarrow$)} Conversely, assume $\lambda\in\rho(A)$.
Then $\sigma(A)$ is closed, so $\rho(A)=\mathbb C\setminus\sigma(A)$ is open. Because
$\lambda\in\mathbb R\cap\rho(A)$, there exists $\varepsilon>0$ such that
\[
(\lambda-\varepsilon,\lambda+\varepsilon)\cap\sigma(A)=\varnothing.
\]
For a self-adjoint operator, the spectral measure $P_A$ is supported on $\sigma(A)$, i.e.,
\[
E\cap\sigma(A)=\varnothing\quad\Longrightarrow\quad P_A(E)=0.
\]
Applying this with $E=(\lambda-\varepsilon,\lambda+\varepsilon)$ yields
\[
P_A\bigl((\lambda-\varepsilon,\lambda+\varepsilon)\bigr)=0.
\]

\medskip
Combining the two directions proves the equivalence, and thus
\[
\sigma(A)=\Bigl\{\lambda\in\mathbb R:\ P_A\bigl((\lambda-\epsilon,\lambda+\epsilon)\bigr)\neq 0,\ \forall\,\epsilon>0\Bigr\}. \]
\end{proof}


    \item Let A be a closed operator and set $|A| = \sqrt{A^*A}.$ Prove that $\||A|f\|=\|Af\|.$ Deduce that $\text{Ker}(A)=\text{Ker}(|A|)=\text{Ran}(|A|)^\perp$ and that 
    $$
    U = \begin{cases}
    g=|A|f \rightarrow  Af& \text{if } g \in \text{Ran}(|A|) \\
    g\rightarrow  0& \text{if } g \in \text{Ker}(|A|)
    \end{cases}
    $$
    extends to a well-defined partial isometry. Conclude that $A=U|A|.$ (This is an extension of the polar decomposition for not necessarily bounded operators)

    \begin{proof}
    \textbf{($\||A|f\|=\|Af\|.$):} So let's derive it be decomposing the norms. Also note that the abs operator is self-adjoint as well.

    \begin{align*}
        \||A|f\|^2&= \langle |A|f , |A|f\rangle \\
        &= \langle f , |A|^2f\rangle \\ 
        &= \langle f , A^*Af\rangle \\ 
        &= \langle Af , Af\rangle \\ 
        &= \|Af\| ^2\\ 
    \end{align*}
    Thus we have shown the equality.

    
    {$\text{(Ker}(A)=\text{Ker}(|A|)=\text{Ran}(|A|)^\perp$):} So for the first equality, note that if $f \in \text{Ker}(|A|)$, then $\||A|f\| = 0$ . Since from the previous statement of the two norms being equal, then $f \in \text{Ker}(|A|)$. We also get a similar result if we assume if $f \in \text{Ker}(A)$.
    
    For the second part of the equality, we use the fact that $A$ is a self-adjoint operator. So a property for any operator is that $ \text{Ran}(|A|)^\perp = \text{Ker}(|A^*|)$. Since A is s.a., we have $ \text{Ran}(|A|)^\perp = \text{Ker}(|A|)$


    (\textbf{Showing $U$ extends to a well-defined partial isometry}):

    Well-Defined: let $f_1,f_2$ be vector such that $|A|f_1=|A|f_2$. Then $$|A|(f_1-f_2) = 0$$ Implying that  $f_1,f_2 \in \text{Ker}(|A|)$. Since $\text{Ker}(|A|) = \text{Ker}(A)$, then $(f_1 - f_2) \in \text{Ker}(A)$. Therefore, $A(f_1 - f_2) = 0 \implies Af_1 = Af_2$. So the mapping $g \mapsto Af$ (where $g=|A|f$) is consistent.
    
    
    (Partial Isometry:) For any vector $g \in \text{Ran}(|A|)$, let $g = |A|f$.
    
    $$
    \|Ug\| = \|U(|A|f)\| = \|Af\|  = \||A|f\| = \|g\|
    $$
    
    Since $\|Ug\| = \|g\|$, the operator $U$ preserves lengths on the range of $|A|$. Since $U$ is an isometry on $\text{Ran}(|A|)$, it extends continuously to the closure $\overline{\text{Ran}(|A|)}$. On the orthogonal complement ($\text{Ran}(|A|)^\perp = \text{Ker}(|A|)$), we defined $U$ to be 0. An operator that is an isometry on a subspace and 0 on its orthogonal complement is the definition of a Partial Isometry.


    ($A=U|A|.$): We first analyze the RHS by letting $g \in \text{Ran}(|A|)$. Then by our definition of $U$, $Ug = Af$. Thus showing the equality. 

    \end{proof}




    
    \item Let $A$ be a self-adjoint operator. Prove that the resolvent operator can be realized as the following representation:

    $$
    R_A(z) = \int_\mathbb{R}\frac{1}{\lambda - z} dP_\lambda.
    $$

    Deduce that the quadratic form $F_\psi(z) = \langle\psi, R_A(z)\psi\rangle$ is a holomorphic function from the upper half plane to itself. (These types of functions are called \textbf{Herglotz} functions and this process is called the \textbf{Borel} transforms of measures.)

    \begin{proof}
        (Resolvent Representation): Our goal is to show that the usual definition of the resolvent operator being $R_A(z) = (A - zI)^{-1}$, with $z\in \rho(A)$, can be rewritten as the intergal in the statement. So to do this, note that $A$ is a s.a. operator. Then by the Spectral Theorem for self adjoint operators, we can rewrite $A$ as

        $$
        A = \int_\mathbb{R}\lambda dP_\lambda.
        $$

        Since $A$ is self-adjoint, its spectrum $\sigma(A)$ is real, so any non-real $z$ belongs to $\rho(A)$. Applying the Borel Functional Calculus, we can define the operator $f(A)$ to be 


        $$
        f(A) =\int_\mathbb{R}f(\lambda) dP_\lambda.
        $$
        To find the integral representation, we look for the function $f(\lambda)$ that corresponds to this operator, We define it to be
        $$f(\lambda) = \frac{1}{\lambda - z}$$
        Therefore, by the functional calculus:$$R_A(z) = (A - zI)^{-1} = \int_{\mathbb{R}} \frac{1}{\lambda - z} \, dP_\lambda$$
        
        (Herglotz): Using the result from Part 1, we substitute the integral representation into the inner product:
        $$F_\psi(z) = \left\langle \psi, \left( \int_{\mathbb{R}} \frac{1}{\lambda - z} dP_\lambda \right) \psi \right\rangle$$
        The scalar measure associated with the vector $\psi$ is defined as $d\mu_\psi(\lambda) = d\langle \psi, P_\lambda \psi \rangle$. This is a positive measure.$$F_\psi(z) = \int_{\mathbb{R}} \frac{1}{\lambda - z} d\mu_\psi(\lambda)$$
        
        Let $z$ be in the upper half-plane, so $z = x + iy$ with $y > 0$. We want to see if the imaginary part of $F_\psi(z)$ is also positive.
        
        Substitute $z$ into the integrand. $$\frac{1}{\lambda - (x + iy)} = \frac{1}{(\lambda - x) - iy}$$
        
        Multiply numerator and denominator by the conjugate $((\lambda - x) + iy)$ to get
        
        $$\frac{(\lambda - x) + iy}{(\lambda - x)^2 + y^2} = \frac{\lambda - x}{(\lambda - x)^2 + y^2} + i \frac{y}{(\lambda - x)^2 + y^2}$$
        
        Now, look at the imaginary part of the integral. $$\text{Im}(F_\psi(z)) = \int_{\mathbb{R}} \text{Im}\left( \frac{1}{\lambda - z} \right) d\mu_\psi(\lambda)$$
        
        $$\text{Im}(F_\psi(z)) = \int_{\mathbb{R}} \frac{y}{(\lambda - x)^2 + y^2} d\mu_\psi(\lambda)$$
        
        From this we have that $y > 0$ (because $z$ is in the upper half-plane), the denominator $(\lambda - x)^2 + y^2$ is always positive, and the measure $d\mu_\psi$ is a positive measure (since $\langle \psi, P_\lambda \psi \rangle$ is a norm squared). Therefore, the integral is positive. $$\text{Im}(F_\psi(z)) > 0$$Since $\text{Im}(z) > 0$ implies $\text{Im}(F_\psi(z)) > 0$, $F_\psi(z)$ maps the upper half plane to itself.
    \end{proof}






\end{enumerate}

\end{document}
